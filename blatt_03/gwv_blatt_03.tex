\documentclass[a4paper]{scrartcl}

% font/encoding packages
\usepackage[utf8]{inputenc}
\usepackage[T1]{fontenc}
\usepackage{lmodern}
\usepackage[ngerman]{babel}
\usepackage[ngerman=ngerman-x-latest]{hyphsubst}

\usepackage{amsmath, amssymb, amsfonts, amsthm}
\usepackage{array}
\usepackage{stmaryrd}
\usepackage{marvosym}
\allowdisplaybreaks
\usepackage[output-decimal-marker={,}]{siunitx}
\usepackage[shortlabels]{enumitem}
\usepackage[section]{placeins}
\usepackage{float}
\usepackage{units}
\usepackage{listings}
\usepackage{pgfplots}
\pgfplotsset{compat=1.12}
\usepackage[hyphens]{url}
\usepackage{hyperref}
\usepackage[newfloat]{minted}

\lstset{
    language=Python,
    numbers=left,
    frame=single,
    basicstyle=\footnotesize\ttfamily,
    otherkeywords={with,as},
}

\newtheorem*{behaupt}{Behauptung}
\newcommand{\gdw}{\Leftrightarrow}

\usepackage{fancyhdr}
\pagestyle{fancy}

\def \blattnr {3}

\lhead{GWV - Blatt {\blattnr}}
\rhead{Billis, Braun, Knapperzbusch, Nikolaisen, Bar}
\cfoot{\thepage}


\title{Grundlagen der Wissensverarbeitung}
\subtitle{Blatt {\blattnr} Hausaufgaben}
\author{
    Fabian Billis (6720351) \\
    Lennart Braun (6523742), \\
    Maximilian Knapperzbusch (6535090) \\
    Laurens Nikolaisen (6527179) \\
    Foo Bar (TODO)
}
\date{zum 2. November 2015}

\begin{document}
\maketitle

\section*{Exercise \blattnr.1: Blind Search}
\begin{enumerate}
    \item
        Das Labyrinth wird als 2D-Array innerhalb eines Objektes der Klasse
        \texttt{Grid} (siehe Listing \ref{lst:search}) gespeichert.  Ein Knoten
        ist ein Indexpaar einer Arrayzelle.  Jeder Knoten ist mit den vertikal
        oder horizontal benachbarte Knoten sind über Kanten verbunden.
        \texttt{Grid} stellt eine \texttt{neighbours} Methode zur Verfügung,
        welche die benachberten Knoten als Liste zurückgibt.  Weiterhin
        speichert \texttt{Grid} die Startposition und enthält Methoden, um zu
        prüfen, ob eine Position ein Ziel oder blockiert ist.

    \item
        Siehe Listing \ref{lst:search}.

    \item
        Siehe Listing \ref{lst:search}.

    \item
        Es muss sich gemerkt werden, welche Felder schon besucht wurden.
        Andernfalls könnte die DFS endlos im Kreis laufen. Die BFS würde
        unnötig viele Pfade in Betracht ziehen.

        Die DFS findet nicht unbedingt den kürzesten Pfad.

        Die Reihenfolge, in der die benachbarten Felder betrachtet werden, hat
        Einfluss auf den zrückgegebenen Pfad. Insbesondere bei der DFS hat
        die Reihenfolge auch Einfluss auf die Länge des gefundenen Pfads.

        \clearpage
    \item
        Die Umgebung könnte nicht berandet sein.
        \begin{verbatim}
                  
       xxx        
       x xxxxx    
   s     x        
       x x  xxxxxx
  xx xxxxx        
      x      g    
      x           
        \end{verbatim}
        Es könnten mehrere Startpunkte existieren.
        \begin{verbatim}
xxxxxxxxxxxxxxxxxxxx
x                  x
x       xxx        x
x       x xxxxx    x
x   s     x        x
x       x x  xxxxxxx
x  xx xxxxx        x
x      x      g  s x
x      x           x
xxxxxxxxxxxxxxxxxxxx
        \end{verbatim}

    \item
        Unbegrenzes Feld:
        Ist klar, dass alle „wichtigen“ Dinge innerhalb des sichtbaren Feldes
        liegen, so kann eine künstliche Begrenzung hinzugefügt werden.  Ist das
        Feld unendlich groß und wird dynamisch generiert, so ist es möglich,
        dass die DFS nicht terminiert.  Eine BFS oder die Iterative Deepening
        Strategie terminieren immer, so lange ein endlicher Pfad zu einem Ziel
        existiert.

        Mehrere Startpunkte:
        Kann von mehreren Knoten aus gestartet werden und die Termination des
        Suchalgorithmus ist garantiert, so kann die Suche nacheinander von
        allen Startknoten aus ausgeführt werden, bis ein Ziel gefunden wurde.
        Ist die Termination nicht garantiert, so müssen die Suchen nebenläufig
        ausgeführt werden. In den Instanzen der Suchen werden abwechselnd
        jeweils eine begrenzte Anzahl an Schritten ausgeführt. Dies garantiert
        Termination, falls eine der Suchen erfolgreich ist.

\end{enumerate}

\clearpage

%\lstinputlisting[%
%    caption=search.py,
%    label=lst:search,
%]{search.py}
\captionof{listing}{search.py \label{lst:search}}
\inputminted[%
    linenos,
    frame=lines,
]{python}{search.py}

\end{document}
