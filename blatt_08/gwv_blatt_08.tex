\documentclass[a4paper]{scrartcl}

% font/encoding packages
\usepackage[utf8]{inputenc}
\usepackage[T1]{fontenc}
\usepackage{lmodern}
\usepackage[ngerman]{babel}
\usepackage[ngerman=ngerman-x-latest]{hyphsubst}

\usepackage{amsmath, amssymb, amsfonts, amsthm}
\usepackage{array}
\usepackage{stmaryrd}
\usepackage{marvosym}
\allowdisplaybreaks
\usepackage[output-decimal-marker={,}]{siunitx}
\usepackage[shortlabels]{enumitem}
\usepackage[section]{placeins}
\usepackage{float}
\usepackage{units}
\usepackage{listings}
\usepackage{pgfplots}
\pgfplotsset{compat=1.12}
\usepackage[hyphens]{url}
\usepackage{hyperref}
\usepackage{pdflscape}

\usepackage{tikz}
\usetikzlibrary{arrows,automata}
\usetikzlibrary{graphs}
\usetikzlibrary{shapes}
\usepackage{verbatim}

%\usepackage[newfloat]{minted}

\lstset{
    language=Python,
    numbers=left,
    frame=single,
    basicstyle=\footnotesize\ttfamily,
    otherkeywords={with,as},
}

\newtheorem*{behaupt}{Behauptung}
\newcommand{\gdw}{\Leftrightarrow}
\newcommand{\prob}{\mathbb{P}}

\usepackage{fancyhdr}
\pagestyle{fancy}

\def \blattnr {8}

\lhead{GWV - Blatt {\blattnr}}
\rhead{Billis, Braun, Knapperzbusch, Nikolaisen}
\cfoot{\thepage}


\title{Grundlagen der Wissensverarbeitung}
\subtitle{Blatt {\blattnr} Hausaufgaben}
\author{
    Fabian Billis (6720351) \\
    Lennart Braun (6523742), \\
    Maximilian Knapperzbusch (6535090) \\
    Laurens Nikolaisen (6527179) \\
}
\date{zum 7. Dezember 2015}

\begin{document}
\maketitle

\section*{Exercise \blattnr.2: Language Modelling}

\begin{itemize}
    \item Siehe \texttt{ticker.py}
        \begin{tiny}
\begin{verbatim}
$ ./ticker.py the 15 3
# found cached marcov chain
the tories out , tactical voting wizard oder stop hague die gegen die tories eingestellten

the art of computer programming ( taocp ) von donald e. knuth sind jetzt online

the bristol group deutschland gmbh gemeinsam mit dem schauspieler patrick stewart ( " captain picard

the perfect storm ( warner ) sowie the patriot ( columbia tristar ) sein .

the microsoft file : the secret case against bill gates ( rang 7 ) ,

the bat kann sogar für lan-netzwerke den lokalen mail-server ersetzen . ein zusatzmodul lokalisiert das

the gap " widmen , die man mit dem handy , wie es robertson versichert

the internet and state control in authoritarian regimes : china , cuba and the counterrevolution

the case when their recipient has communicated his/her address for this purpose or in full

the millenium " von hps enterprise computing group zielt besonders auf web-dienstleistungsanbieter . diese unternehmen
\end{verbatim}
\end{tiny}
    \item
        Wortfolgen, die in echten Texten häufig sind, tauchen auf (nach Konstruktion).
        Inhaltlich und grammatikalisch sind die generierten Wortfolgen nur bedingt richtig.
        Letzteres aber für diesen Anwendungszweck unter Umständen weniger
        wichtig, da Eilmeldungen bzw.  Überschriften nicht immer den
        grammatikalischen Regeln entsprechen.

    \item Siehe \texttt{ticker.py}
\end{itemize}

\section*{Exercise \blattnr.3: Diagnosis (cont.)}

\begin{figure}[h]
    \centering
    \begin{tikzpicture}[%
            auto,
            node distance=1cm,
            draw,
            scale=3,
            component/.style = {
                draw,
                ellipse,
                align=center,
            },
            edge/.style = {
                ->,
                >=stealth',
            },
        ]
        \node [component] (ik)  at (0, 3) {ignition \\ key};
        \node [component] (bat) at (1, 3) {battery};
        \node [component] (st)  at (0, 1) {starter};
        \node [component] (fl)  at (1, 1) {filter};
        \node [component] (efr) at (2, 2) {electronic \\ fuel \\ regulation};
        \node [component] (ft)  at (3, 2) {fuel tank};
        \node [component] (fp)  at (2, 1) {fuel pump};
        \node [component] (en)  at (1, 0) {engine};

        \draw [edge] (ik) -- (st);
        \draw [edge] (ik) -- (efr);
        \draw [edge] (bat) -- (st);
        \draw [edge] (bat) -- (efr);
        \draw [edge] (ft) -- (fp);
        \draw [edge] (efr) -- (fp);
        \draw [edge] (st) -- (en);
        \draw [edge] (fl) -- (en);
        \draw [edge] (fp) -- (en);
    \end{tikzpicture}
    \caption{Belief Network zum Auto}
\end{figure}

Da die Batterie von keiner Komponente abhängig ist, gibt es eine
Wahrscheinlichkeit von \num{0,9}, dass die Batterie funktioniert. Mit Hilfe des
Satzes der totalen Wahrscheinlichkeit lässt sich bestimmen, mit welcher
Wahrscheinlichkeit die Bauteile funktionieren werden.

Es ergeben sich die folgenden Wahrscheinlichkeiten:
\begin{align*}
    \prob(battery) &= \num{0,9} \\
    \begin{split}
        \prob(starter)
        &= \prob(st\ |\ ik \cap bat) \cdot \prob(ik \cap bat) \\
        &= \prob(st\ |\ ik \cap bat) \cdot \prob(ik) \cdot \prob(bat) \\
        &=\num{0,9}^3 \\
        &= \prob(efr)
    \end{split} \\
    \begin{split}
        \prob(engine)
        &= \prob(en\ |\ st \cap fl \cap fp) \cdot \prob(st \cap fl \cap fp) \\
        &= \prob(en\ |\ st \cap fl \cap fp) \cdot \prob(st) \cdot
           \prob(fl) \cdot \prob(fp) \\
        &= \prob(en\ |\ st \cap fl \cap fp) \cdot \prob(st) \cdot
           \prob(fl) \cdot \prob(fp\ |\ efr \cap ft) \cdot \prob(efr \cap ft) \\
        &= \prob(en\ |\ st \cap fl \cap fp) \cdot \prob(st) \cdot
           \prob(fl) \cdot \prob(fp\ |\ efr \cap ft) \cdot \prob(efr) \cdot \prob(ft) \\
        &= \num{0,9} \cdot \num{0,9}^3 \cdot \num{0,9} \cdot \num{0,9} \cdot
           \num{0,9}^3 \cdot \num{0,9} \\
        &= \num{0,9}^{10}
    \end{split} \\
    \begin{split}
        \prob(engine\ |\ fuel\_pump)
        &= \frac{\prob(en \cap fp)}{\prob(fp)} \\
        &= \frac{\prob(en)}{\prob(fp)} \\
        &= \frac{\prob(en)}{\prob(fp\ |\ efr \cap ft) \cdot \prob(efr \cap ft)} \\
        &= \frac{\prob(en)}{\prob(fp\ |\ efr \cap ft) \cdot \prob(efr) \cdot \prob(ft)} \\
        &= \frac{\num{0,9}^{10}}{\num{0,9} \cdot \num{0,9}^3 \cdot \num{0,9}} \\
        &= \num{0,9}^5 \\
    \end{split}
\end{align*}


\section*{Exercise \blattnr.4: Bayesian Probabilities}

Wir verwenden die folgenden Ereignisse:
\begin{align*}
    sm &\quad \text{Die Person ist ein Schmuggler.} \\
    bark &\quad \text{Der Hund bellt.} \\
    sweat &\quad \text{Die Person schwitzt.} \\
    fever &\quad \text{Die Person hat Fieber.}
\end{align*}
Das Gegenereignis zu einem Ereignis $A$ wird mit $A^C$ bezeichnet.

Aus dem Text ergibt sich die Wahrscheinlichkeiten:
\begin{align*}
    \prob(sm) &= \num{0,01} \\
    \prob(bark\ |\ sm) &= \num{0,8} \\
    \prob(bark\ |\ sm^C) &= \num{0,05} \\
    \prob(sweat\ |\ sm^C \cap fever^C) &= \num{0} \\
    \prob(sweat\ |\ sm \cap fever^C) &= \num{0.4} \\
    \prob(sweat\ |\ sm \cap fever) &= \num{0.8} \\
    \prob(sweat\ |\ sm^C \cap fever) &= \num{0.6} \\
    \prob(fever) &= \num{0,013} \\
\end{align*}

\begin{figure}[h]
    \centering
    \begin{tikzpicture}[%
            auto,
            node distance=1cm,
            draw,
            scale=3,
            component/.style = {
                draw,
                ellipse,
                align=center,
            },
            edge/.style = {
                ->,
                >=stealth',
            },
        ]
        \node [component] (sm) at (0, 1) {smuggler};
        \node [component] (fv) at (1, 1) {fever};
        \node [component] (ba) at (0, 0) {dog barks};
        \node [component] (sw) at (1, 0) {sweat};

        \draw [edge] (sm) -- (ba);
        \draw [edge] (sm) -- (sw);
        \draw [edge] (fv) -- (sw);
    \end{tikzpicture}
    \caption{Belief Network zum Schmugler}
\end{figure}

Beispiel für \emph{explaining away} im Belief Network:
Seien $a_1,b_1$ die bedingten Wahrscheinlichkeiten dafür, dass eine Person ein
Schmuggler ist bzw. dass eine Person Fieber hat, gegeben, dass sie schwitzt.
\begin{align*}
    \prob(smuggler\ |\ sweat) &= a_1 \\
    \prob(fever\ |\ sweat) &= b_1
\end{align*}
Wird nun zusätzlich der bellender Hund beobachtet, so ergeben sich die
Wahrscheinlichkeiten $a_2, b_2$.
\begin{align*}
    \prob(smuggler\ |\ sweat \cap bark) &= a_2 \\
    \prob(fever\ |\ sweat \cap bark) &= b_2
\end{align*}
Die Wahrscheinlichkeit für einen Schmuggler ist gestiegen ($a_2 > a_1$), die
für das Fieber gesunken ($b_2 < b_1$). Das beobachtete Schwitzen wurde durch
das nun wahrscheinlichere Schmuggeln weg erklärt.

Wir nehmen an, dass das Ereignis $fever$ von dem Ereignis $sm$ stochastisch
unabhängig ist.  Mit dem Satz von der totalen Wahrscheinlichkeit ergeben sich:
\begin{align*}
    \begin{split}
        \prob(bark) &= \prob(bark\ |\ sm) \cdot \prob(sm) +
                       \prob(bark\ |\ sm^C) \cdot \prob(sm^C) \\
                    &= \prob(bark\ |\ sm) \cdot \prob(sm) +
                       \prob(bark\ |\ sm^C) \cdot (1 - \prob(sm)) \\
                    &= \num{0,8} \cdot \num{0,01} +
                       \num{0,05} \cdot \num{0,99} \\
                    &= \num{0,0575}
    \end{split} \\
    \begin{split}
        \prob(sweat) &= \prob(sweat\ |\ sm \cap fever) \cdot
                        \prob(sm \cap fever) \\
                     &+ \prob(sweat\ |\ sm \cap fever^C) \cdot
                        \prob(sm \cap fever^C) \\
                     &+ \prob(sweat\ |\ sm^C \cap fever) \cdot
                        \prob(sm^C \cap fever) \\
                     &+ \prob(sweat\ |\ sm^C \cap fever^C) \cdot
                        \prob(sm^C \cap fever^C) \\
                     &= \prob(sweat\ |\ sm \cap fever) \cdot
                        \prob(sm) \cdot \prob(fever) \\
                     &+ \prob(sweat\ |\ sm \cap fever^C) \cdot
                        \prob(sm) \cdot (1 - \prob(fever)) \\
                     &+ \prob(sweat\ |\ sm^C \cap fever) \cdot
                        (1 - \prob(sm)) \cdot \prob(fever) \\
                     &+ \prob(sweat\ |\ sm^C \cap fever^C) \cdot
                        (1 - \prob(sm)) \cdot (1 - \prob(fever)) \\
                        &= \num{0,8} \cdot
                        \num{0,01} \cdot \num{0,013} \\
                        &+ \num{0,4} \cdot
                        \num{0,01} \cdot \num{0,987} \\
                        &+ \num{0,6} \cdot
                        \num{0,99} \cdot \num{0,013} \\
                        &+ \num{0} \cdot
                        \num{0,99} \cdot \num{0,987} \\
                     &= \num{0.011774}
    \end{split}
\end{align*}

\begin{itemize}
    \item
        Schmuggler, gegeben der Hund bellt.
        (Wir verwenden den Satz von Bayes.)
        \begin{equation*}
            \begin{split}
                \prob(sm\ |\ bark)
                &= \frac{\prob(sm) \cdot \prob(bark\ |\ sm)}{\prob(bark)} \\
                &= \frac{\num{0,01} \cdot \num{0,8}}{\num{0,0575}} \\
                &= \num{0.139}
            \end{split}
        \end{equation*}

    \item Mensch schwitzt.
        \begin{equation*}
            \begin{split}
                \prob(sweat) = \num{0.011774}
                \qquad \text{s.o.}
            \end{split}
        \end{equation*}

    \item Schmuggler, gegeben, dass Mensch schwitzt und der Hund bellt.
        \begin{equation*}
            \begin{split}
                \prob(sm\ |\ sweat \cap bark)
                &= \frac{\prob(sweat \cap bark\ |\ sm) \cdot \prob(sm)}
                        {\prob(sweat \cap bark)} \\
                &= \frac{\prob(sweat\ |\ sm) \cdot \prob(bark\ |\ sm) \cdot
                   \prob(sm)}{\prob(sweat) \cdot \prob(bark)} \\
                &= \frac{\frac{\prob(sm\ |\ sweat) \cdot \prob(sweat)}
                              {\prob(sm)}
                         \cdot \prob(bark\ |\ sm) \cdot \prob(sm)}
                        {\prob(sweat) \cdot \prob(bark)} \\
                &= \frac{\prob(sm\ |\ sweat) \cdot \prob(bark\ |\ sm)}
                        {\prob(bark)} \\
                &= \frac{(\prob(sm \cap fever\ |\ sweat) +
                          \prob(sm \cap fever^C\ |\ sweat)) \cdot
                         \prob(bark\ |\ sm)}
                        {\prob(bark)} \\
                &= \frac{(\frac{\prob(sweat\ |\ sm \cap fever) \cdot \prob(sm \cap fever)}{\prob(sweat)} +
                          \frac{\prob(sweat\ |\ sm \cap fever^C) \cdot \prob(sm \cap fever^C)}{\prob(sweat)})}
                        {\prob(bark)} \\
                &\cdot \prob(bark\ |\ sm) \\
                &= \frac{(\prob(sweat\ |\ sm \cap fever) \cdot \prob(fever) +
                          \prob(sweat\ |\ sm \cap fever^C) \cdot \prob(fever^C))}
                          {\prob(sweat) \cdot \prob(bark)} \\
                &\cdot \prob(bark\ |\ sm) \cdot \prob(sm)\\
                &= \frac{(\num{0,8} \cdot \num{0,013} + \num{0,4} \cdot \num{0,987})}
                        {\num{0,011774} \cdot \num{0,0575}}
                   \cdot \num{0,8} \cdot \num{0,01} \\
                &= \num{4,788}\ \lightning
            \end{split}
        \end{equation*}
        Hier ist etwas schief gegangen.
\end{itemize}

\end{document}
