\documentclass[a4paper]{scrartcl}

% font/encoding packages
\usepackage[utf8]{inputenc}
\usepackage[T1]{fontenc}
\usepackage{lmodern}
\usepackage[ngerman]{babel}
\usepackage[ngerman=ngerman-x-latest]{hyphsubst}

\usepackage{amsmath, amssymb, amsfonts, amsthm}
\usepackage{array}
\usepackage{stmaryrd}
\usepackage{marvosym}
\allowdisplaybreaks
\usepackage[output-decimal-marker={,}]{siunitx}
\usepackage[shortlabels]{enumitem}
\usepackage[section]{placeins}
\usepackage{float}
\usepackage{units}
\usepackage{listings}
\usepackage{pgfplots}
\pgfplotsset{compat=1.12}
\usepackage[hyphens]{url}
\usepackage{hyperref}
\usepackage{pdflscape}

\usepackage{tikz}
\usetikzlibrary{arrows,automata}
\usetikzlibrary{graphs}
\usetikzlibrary{shapes}
\usepackage{verbatim}

%\usepackage[newfloat]{minted}

\lstset{
    language=Python,
    numbers=left,
    frame=single,
    basicstyle=\footnotesize\ttfamily,
    otherkeywords={with,as},
}

\newtheorem*{behaupt}{Behauptung}
\newcommand{\gdw}{\Leftrightarrow}
\newcommand{\prob}{\mathbb{P}}

\usepackage{fancyhdr}
\pagestyle{fancy}

\def \blattnr {8}

\lhead{GWV - Blatt {\blattnr}}
\rhead{Billis, Braun, Knapperzbusch, Nikolaisen}
\cfoot{\thepage}


\title{Grundlagen der Wissensverarbeitung}
\subtitle{Blatt {\blattnr} Hausaufgaben}
\author{
    Fabian Billis (6720351) \\
    Lennart Braun (6523742), \\
    Maximilian Knapperzbusch (6535090) \\
    Laurens Nikolaisen (6527179) \\
}
\date{zum 7. Dezember 2015}

\begin{document}
\maketitle

\section*{Exercise \blattnr.2: Language Modelling}

\section*{Exercise \blattnr.3: Diagnosis (cont.)}

\begin{figure}[h]
    \centering
    \begin{tikzpicture}[%
            auto,
            node distance=1cm,
            draw,
            scale=3,
            component/.style = {
                draw,
                ellipse,
                align=center,
            },
            edge/.style = {
                ->,
                >=stealth',
            },
        ]
        \node [component] (ik)  at (0, 3) {ignition \\ key};
        \node [component] (bat) at (1, 3) {battery};
        \node [component] (st)  at (0, 1) {starter};
        \node [component] (fl)  at (1, 1) {filter};
        \node [component] (efr) at (2, 2) {electronic \\ fuel \\ regulation};
        \node [component] (ft)  at (3, 2) {fuel tank};
        \node [component] (fp)  at (2, 1) {fuel pump};
        \node [component] (en)  at (1, 0) {engine};

        \draw [edge] (ik) -- (st);
        \draw [edge] (ik) -- (efr);
        \draw [edge] (bat) -- (st);
        \draw [edge] (bat) -- (efr);
        \draw [edge] (ft) -- (fp);
        \draw [edge] (efr) -- (fp);
        \draw [edge] (st) -- (en);
        \draw [edge] (fl) -- (en);
        \draw [edge] (fp) -- (en);
    \end{tikzpicture}
    \caption{Belief Network zum Auto}
\end{figure}
Da die Batterie von keiner Komponente abhängig ist, gibt es eine Wahrscheinlichkeit von $0,9$, dass
die Battiere funktioniert. Für den Starter und die Engine muss man den Satz der totalen 
Wahrscheinlichkeit anwenden, um zu bestimmen, mit welcher Wahrscheinlichkeit die Bauteile funktionieren
werden. Bei der letzten Fragestellung müssen wir den Satz von Bayes verwenden. In diesem Fall ist die 
Engine von der fuelpump ab abhängig.
\begin{align*}
    \prob(battery) &= \num{0,9} \\
    \begin{split}
        \prob(starter)
        &= \prob(st\ |\ ik \cap bat) \cdot \prob(ik \cap bat) \\
        &= \prob(st\ |\ ik \cap bat) \cdot \prob(ik) \cdot \prob(bat) \\
        &=\num{0,9}^3 \\
        &= \prob(efr)
    \end{split} \\
    \begin{split}
        \prob(engine)
        &= \prob(en\ |\ st \cap fl \cap fp) \cdot \prob(st \cap fl \cap fp) \\
        &= \prob(en\ |\ st \cap fl \cap fp) \cdot \prob(st) \cdot \prob(fl) \cdot \prob(fp) \\
        &= \prob(en\ |\ st \cap fl \cap fp) \cdot \prob(st) \cdot \prob(fl) \cdot \prob(fp\ |\ efr \cap ft) \cdot \prob(efr \cap ft) \\
        &= \prob(en\ |\ st \cap fl \cap fp) \cdot \prob(st) \cdot \prob(fl) \cdot \prob(fp\ |\ efr \cap ft) \cdot \prob(efr) \cdot \prob(ft) \\
        &= \num{0,9} \cdot \num{0,9}^3 \cdot \num{0,9} \cdot \num{0,9} \cdot \num{0,9}^3 \cdot \num{0,9} \\
        &= \num{0,9}^{10}
    \end{split} \\
    \begin{split}
        \prob(engine\ |\ fuel\_pump)
        &= \frac{\prob(en \cap fp)}{\prob(fp)} \\
        &= \frac{\prob(en)}{\prob(fp)} \\
        &= \frac{\prob(en)}{\prob(fp\ |\ efr \cap ft) \cdot \prob(efr \cap ft)} \\
        &= \frac{\prob(en)}{\prob(fp\ |\ efr \cap ft) \cdot \prob(efr) \cdot \prob(ft)} \\
        &= \frac{\num{0,9}^{10}}{\num{0,9} \cdot \num{0,9}^3 \cdot \num{0,9}} \\
        &= \num{0,9}^5 \\
    \end{split}
\end{align*}


\section*{Exercise \blattnr.4: Bayesian Probabilities}

Wir verwenden die folgenden Ereignisse:
\begin{align*}
    sm &\quad \text{Die Person ist ein Schmuggler.} \\
    bark &\quad \text{Der Hund bellt.} \\
    sweat &\quad \text{Die Person schwitzt.} \\
    fever &\quad \text{Die Person hat Fieber.}
\end{align*}
Das Gegenereignis zu einem Ereignis $A$ wird mit $A^C$ bezeichnet.

Aus dem Text ergibt sich die Wahrscheinlichkeiten:
\begin{align*}
    \prob(sm) &= \num{0,01} \\
    \prob(bark\ |\ sm) &= \num{0,8} \\
    \prob(bark\ |\ sm^C) &= \num{0,05} \\
    \prob(sweat\ |\ sm^C \cap fever^C) &= \num{0} \\
    \prob(sweat\ |\ sm \cap fever^C) &= \num{0.4} \\
    \prob(sweat\ |\ sm \cap fever) &= \num{0.8} \\
    \prob(sweat\ |\ sm^C \cap fever) &= \num{0.6} \\
    \prob(fever) &= \num{0,013} \\
\end{align*}

TODO: network

Wir nehmen an, dass das Ereignis $fever$ von dem Ereignis $sm$ stochastisch
unabhängig ist.  Mit dem Satz von der totalen Wahrscheinlichkeit ergibt sich:
\begin{align*}
    \begin{split}
        \prob(bark) &= \prob(bark\ |\ sm) \cdot \prob(sm) +
                       \prob(bark\ |\ sm^C) \cdot \prob(sm^C) \\
                    &= \prob(bark\ |\ sm) \cdot \prob(sm) +
                       \prob(bark\ |\ sm^C) \cdot (1 - \prob(sm)) \\
                    &= \num{0,8} \cdot \num{0,01} +
                       \num{0,05} \cdot \num{0,99} \\
                    &= \num{0,0575}
    \end{split} \\
    \begin{split}
        \prob(sweat) &= \prob(sweat\ |\ sm \cap fever) \cdot
                        \prob(sm \cap fever) \\
                     &+ \prob(sweat\ |\ sm \cap fever^C) \cdot
                        \prob(sm \cap fever^C) \\
                     &+ \prob(sweat\ |\ sm^C \cap fever) \cdot
                        \prob(sm^C \cap fever) \\
                     &+ \prob(sweat\ |\ sm^C \cap fever^C) \cdot
                        \prob(sm^C \cap fever^C) \\
                     &= \prob(sweat\ |\ sm \cap fever) \cdot
                        \prob(sm) \cdot \prob(fever) \\
                     &+ \prob(sweat\ |\ sm \cap fever^C) \cdot
                        \prob(sm) \cdot (1 - \prob(fever)) \\
                     &+ \prob(sweat\ |\ sm^C \cap fever) \cdot
                        (1 - \prob(sm)) \cdot \prob(fever) \\
                     &+ \prob(sweat\ |\ sm^C \cap fever^C) \cdot
                        (1 - \prob(sm)) \cdot (1 - \prob(fever)) \\
                        &= \num{0,8} \cdot
                        \num{0,01} \cdot \num{0,013} \\
                        &+ \num{0,4} \cdot
                        \num{0,01} \cdot \num{0,987} \\
                        &+ \num{0,6} \cdot
                        \num{0,99} \cdot \num{0,013} \\
                        &+ \num{0} \cdot
                        \num{0,99} \cdot \num{0,987} \\
                     &= \num{0.011774}
    \end{split}
\end{align*}

TODO: example of explaining away in the network

Wir verwenden den Satz von Bayes.
\begin{equation*}
    \begin{split}
        \prob(sm\ |\ bark)
        &= \frac{\prob(sm) \cdot \prob(bark\ |\ sm)}{\prob(bark)} \\
        &= \frac{\num{0,01} \cdot \num{0,8}}{\num{0,0575}} \\
        &= \num{0.1391304347826087}
    \end{split}
\end{equation*}
\begin{equation*}
    \begin{split}
        \prob(sweat) = \num{0.011774}
        \qquad \text{s.o.}
    \end{split}
\end{equation*}
\begin{equation*}
    \begin{split}
        \prob(sm\ |\ sweat \cap bark) = TODO
        \qquad \text{s.o.}
    \end{split}
\end{equation*}

\end{document}
