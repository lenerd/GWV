\documentclass[a4paper]{scrartcl}

% font/encoding packages
\usepackage[utf8]{inputenc}
\usepackage[T1]{fontenc}
\usepackage{lmodern}
\usepackage[ngerman]{babel}
\usepackage[ngerman=ngerman-x-latest]{hyphsubst}

\usepackage{amsmath, amssymb, amsfonts, amsthm}
\usepackage{array}
\usepackage{stmaryrd}
\usepackage{marvosym}
\allowdisplaybreaks
\usepackage[output-decimal-marker={,}]{siunitx}
\usepackage[shortlabels]{enumitem}
\usepackage[section]{placeins}
\usepackage{float}
\usepackage{units}
\usepackage{listings}
\usepackage{pgfplots}
\pgfplotsset{compat=1.12}
\usepackage[hyphens]{url}
\usepackage{hyperref}
\usepackage{pdflscape}

\usepackage{tikz}
\usetikzlibrary{arrows,automata}
\usepackage{verbatim}

%\usepackage[newfloat]{minted}

\lstset{
    language=Python,
    numbers=left,
    frame=single,
    basicstyle=\footnotesize\ttfamily,
    otherkeywords={with,as},
}

\newtheorem*{behaupt}{Behauptung}
\newcommand{\gdw}{\Leftrightarrow}

\usepackage{fancyhdr}
\pagestyle{fancy}

\def \blattnr {7}

\lhead{GWV - Blatt {\blattnr}}
\rhead{Billis, Braun, Knapperzbusch, Nikolaisen}
\cfoot{\thepage}


\title{Grundlagen der Wissensverarbeitung}
\subtitle{Blatt {\blattnr} Hausaufgaben}
\author{
    Fabian Billis (6720351) \\
    Lennart Braun (6523742), \\
    Maximilian Knapperzbusch (6535090) \\
    Laurens Nikolaisen (6527179) \\
}
\date{zum 30. November 2015}

\begin{document}
\maketitle

\section*{Exercise \blattnr.2: CSI Stellingen}

Assumables $A$:
\begin{align*}
    gardener\_worked & \\
    butler\_worked &
\end{align*}
Observations:
\begin{align*}
    gardener\_clean & \\
    butler\_dirty &
\end{align*}
Rules:
\begin{align*}
    gardener\_dirty &\leftarrow gardener\_worked \\
    butler\_dirty &\leftarrow butler\_worked
\end{align*}
Integrity Constraints:
\begin{align*}
    false &\leftarrow gardener\_dirty \land gardener\_clean \\
    false &\leftarrow butler\_dirty \land butler\_clean
\end{align*}
Unsere Knowledge Base $KB$ bestehe aus den Observations, den Rules und den
Integrity Constraints.  Nun fragen wir nach $false$.  Mit Hilfe des Top-Down
Verfahrens können wir den folgenden Baum erstellen.  Die gestrichelten Kanten
existieren nur, wenn $KB$ mit einer Menge $c \in \{ C \subseteq A \ |\
g\_worked \in C\}$ vereinigen. Konflikte sind also $\{g\_worked\}$ und
$\{g\_worked, b\_worked\}$, wobei ersterer ein minimaler Konflikt ist.  Eine
Diagnose ist eine Menge von Annahmen, so dass aus jedem (minimalen) Konflikt
mindestens eine Annahme in der Diagnose enthalten ist. In diesem Fall ist die
minimale Diagnose $\{g\_worked\}$. Der Gärtner kann also nicht gearbeitet
haben.
\begin{figure}[h]
    \begin{tikzpicture}[%
            level/.style={sibling distance=80mm/#1},
        ]
        \node (query) {$yes \leftarrow false$}
            child {
                node {$yes \leftarrow g\_dirty \land g\_clean$}
                child {
                    node {$yes \leftarrow g\_dirty$}
                    child {
                        node {$yes \leftarrow g\_worked$}
                        child [dashed] {
                            node {$yes \leftarrow$}
                        }
                    }
                }
                child {
                    node {$yes \leftarrow g\_worked \land g\_clean$}
                    child {
                        node {$yes \leftarrow g\_worked$}
                        child [dashed] {
                            node {$yes \leftarrow$}
                        }
                    }
                }
            }
            child {
                node {$yes \leftarrow b\_dirty \land b\_clean$}
                child {
                    node {$yes \leftarrow b\_clean$}
                }
            }
        ;
    \end{tikzpicture}
    \caption{Suchgraph für die Top-Down Ableitung}
\end{figure}

\section*{Exercise \blattnr.3: Diagnosis}


Wir verwenden die folgenden Abkürzungen:
\begin{align*}
    bat &= battery \\
    ik &= ignition\_key \\
    efr &= electronic\_fuel\_regulation \\
    st &= starter \\
    en &= engine \\
    fl &= filter \\
    fp &= fuel\_pump \\
    ft &= fuel\_tank \\
\end{align*}
Es wird davon ausgegangen, dass alle Komponenten funktionstüchtig sind ($ok\_\ast$). \\
\textbf{Assumables:}
\begin{align*}
    ok\_battery & \\
    ok\_ignition\_key & \\
    ok\_electronic\_fuel\_regulation & \\
    ok\_starter & \\
    ok\_engine & \\
    ok\_filter & \\
    ok\_fuel\_pump & \\
    ok\_fuel\_tank &
\end{align*}
Läuft eine Komponente bzw. erfüllt sie ihren Zweck, so gilt $on\_\ast$. \\
\textbf{Rules:}
\begin{align*}
    on\_bat &\leftarrow ok\_bat \\
    on\_ik &\leftarrow ok\_ik \\
    on\_efr &\leftarrow ok\_efr \land on\_bat \land on\_ik \\
    on\_st &\leftarrow ok\_st \land on\_ik \land on\_bat \\
    on\_en &\leftarrow ok\_en \land on\_st \land on\_fl \land on\_fp \\
    on\_fl &\leftarrow ok\_fl \\
    on\_fp &\leftarrow ok\_fp \land on\_ft \land on\_efr \\
    on\_ft &\leftarrow ok\_ft \\
    noise\_1 &\leftarrow on\_st \\
    noise\_2 &\leftarrow on\_fp \\
    noise\_3 &\leftarrow on\_en
\end{align*}
Ein Geräusch kann nicht gleichzeitig zu hören sein und nicht zu hören sein. \\
\textbf{Integrity Constraints:}
\begin{align*}
    false &\leftarrow noise\_1 \land no\_noise\_1 \\
    false &\leftarrow noise\_2 \land no\_noise\_2 \\
    false &\leftarrow noise\_3 \land no\_noise\_3
\end{align*}

Nun kann mittels des Top-Down Algorithmus ein Ableitungsbaum zu dem Query
$false$ erstellt werden (siehe Abb. \ref{fig:2} auf Seite \pageref{fig:2}).
Dabei wird die Knowledge Base aus den Rules, den Integrity Constraings und den
Observations ($\{no\_noise\_1, no\_noise\_2, no\_noise\_3\}$, $\{no\_noise\_2,
no\_noise\_3\}$, $\{no\_noise\_1, no\_noise\_3\}$ oder $\{no\_noise\_3\}$)
verwendet.

Damit der Baum noch einigermaßen übersichtlich bleibt, haben wir einige
Vereinfachungen vorgenommen. Da die Reihenfolge von Ableitungen keinen
wesentlichen Unterschied macht, wurden nur die deri Hauptzweige dargestellt.
Weiter wurden Schritte zusammengefasse, wenn nacheinander mehrere $on\_\ast$
durch die entsprechenden $ok\_\ast$ ersetzt wurden.

Wenn das $i$-te Geräusch nicht gehört wurde, so kann die letzte (gestrichelte)
Ableitung im $i$-ten Ast des Baumes durchgeführt werden.

\begin{enumerate}
    \item Es wurde kein Geräusch gehört.
        \begin{itemize}
            \item Die Konflikte sind
                $\{ok\_st, ok\_ik, ok\_bat\}$,
                $\{ok\_fp, ok\_ft, ok\_efr, ok\_bat, ok\_ik\}$ und
                $\{ok\_en, ok\_st, ok\_ik, ok\_bat, ok\_fl,ok\_fp, ok\_ft,
                ok\_efr\}$.
            \item Minimal sind
                $\{ok\_st, ok\_ik, ok\_bat\}$ und
                $\{ok\_fp, ok\_ft, ok\_efr, ok\_bat, ok\_ik\}$.
            \item Daraus ergeben sich die minimalen Diagnosen
                $\{ok\_ik\}$, $\{ok\_bat\}$, $\{ok\_st, ok\_fp\}$, $\{ok\_st,
                ok\_ft\}$ und $\{ok\_st, ok\_efr\}$.
        \end{itemize}

    \item Es wurde nur Geräusch 1 gehört.
        \begin{itemize}
            \item Die Konflikte sind
                $\{ok\_fp, ok\_ft, ok\_efr, ok\_bat, ok\_ik\}$ und \\
                $\{ok\_en, ok\_st, ok\_ik, ok\_bat, ok\_fl,ok\_fp, ok\_ft,
                ok\_efr\}$.
            \item Minimal ist
                $\{ok\_fp, ok\_ft, ok\_efr, ok\_bat, ok\_ik\}$.
            \item Daraus ergeben sich die minimalen Diagnosen
                $\{ok\_fp\}$, $\{ok\_ft\}$, $\{ok\_efr\}$, $\{ok\_ik\}$ und
                $\{ok\_bat\}$.
        \end{itemize}

    \item Es wurde nur Geräusch 2 gehört.
        \begin{itemize}
            \item Die Konflikte sind
                $\{ok\_st, ok\_ik, ok\_bat\}$ und \\
                $\{ok\_en, ok\_st, ok\_ik, ok\_bat, ok\_fl,ok\_fp, ok\_ft,
                ok\_efr\}$.
            \item Minimal ist
                $\{ok\_st, ok\_ik, ok\_bat\}$.
            \item Daraus ergeben sich die minimalen Diagnosen
                $\{ok\_st\}$, $\{ok\_ik\}$ und $\{ok\_bat\}$.
        \end{itemize}

    \item Es wurden nur Geräusche 1 und 2 gehört.
        \begin{itemize}
            \item Der Konflikt ist
                $\{ok\_en, ok\_st, ok\_ik, ok\_bat, ok\_fl,ok\_fp, ok\_ft,
                ok\_efr\}$.
            \item Minimal ist
                $\{ok\_en, ok\_st, ok\_ik, ok\_bat, ok\_fl,ok\_fp, ok\_ft,
                ok\_efr\}$.
            \item Daraus ergeben sich die minimalen Diagnosen
                $\{ok\_en\}$, $\{ok\_st\}$, $\{ok\_ik\}$, $\{ok\_bat\}$,
                $\{ok\_fl,ok\_fp\}$, $\{ok\_ft\}$ und $\{ok\_efr\}$.
        \end{itemize}

\end{enumerate}


\begin{landscape}
    \pagestyle{empty}
    %\vspace*{\fill}
    \begin{figure}[h]
        \begin{tikzpicture}[%
                level/.style={
                    sibling distance=48mm/#1,
                    level distance=10mm,
                },
                scale=1.5,
                align=center,
                text width=70mm,
            ]
            \node (query) {$false$}
                child {
                    node {$noise\_1 \land no\_noise\_1$}
                    child {
                        node {$on\_st \land no\_noise\_1$}
                        child {
                            node {$ok\_st \land on\_ik \land on\_bat \land no\_noise\_1$}
                            child  {
                                node {$ok\_st \land ok\_ik \land ok\_bat \land no\_noise\_1$}
                                child [dashed]  {
                                    node {$ok\_st \land ok\_ik \land ok\_bat$}
                                }
                            }
                        }
                    }
                }
                child {
                    node {$noise\_2 \land no\_noise\_2$}
                    child {
                        node {$on\_fp \land no\_noise\_2$}
                        child {
                            node {$ok\_fp \land on\_ft \land on\_efr \land no\_noise\_2$}
                            child {
                                node {$ok\_fp \land ok\_ft \land on\_efr \land no\_noise\_2$}
                                child {
                                    node {$ok\_fp \land ok\_ft \land ok\_efr \land on\_bat \land on\_ik \land no\_noise\_2$}
                                    child {
                                        node {$ok\_fp \land ok\_ft \land ok\_efr \land ok\_bat \land ok\_ik \land no\_noise\_2$}
                                        child [dashed] {
                                            node {$ok\_fp \land ok\_ft \land ok\_efr \land ok\_bat \land ok\_ik$}
                                        }
                                    }
                                }
                            }
                        }
                    }
                }
                child {
                    node {$noise\_3 \land no\_noise\_3$}
                    child {
                        node {$on\_en \land no\_noise\_3$}
                        child {
                            node {$ok\_en \land on\_st \land on\_fl \land on\_fp \land no\_noise\_3$}
                            child {
                                node {$ok\_en \land ok\_st \land on\_ik \land on\_bat \land on\_fl \land on\_fp \land no\_noise\_3$}
                                child {
                                    node {$ok\_en \land ok\_st \land on\_ik \land on\_bat \land on\_fl \land on\_fp \land no\_noise\_3$}
                                    child {
                                        node {$ok\_en \land ok\_st \land ok\_ik \land ok\_bat \land ok\_fl \land ok\_fp \land on\_ft \land on\_efr \land no\_noise\_3$}
                                        child {
                                            node {$ok\_en \land ok\_st \land ok\_ik \land ok\_bat \land ok\_fl \land ok\_fp \land ok\_ft \land on\_efr \land no\_noise\_3$}
                                            child {
                                                node {$ok\_en \land ok\_st \land ok\_ik \land ok\_bat \land ok\_fl \land ok\_fp \land ok\_ft \land ok\_efr \land on\_bat \land on\_ik \land no\_noise\_3$}
                                                child {
                                                    node {$ok\_en \land ok\_st \land ok\_ik \land ok\_bat \land ok\_fl \land ok\_fp \land ok\_ft \land ok\_efr \land no\_noise\_3$}
                                                    child [dashed] {
                                                        node {$ok\_en \land ok\_st \land ok\_ik \land ok\_bat \land ok\_fl \land ok\_fp \land ok\_ft \land ok\_efr$}
                                                    }
                                                }
                                            }
                                        }
                                    }
                                }
                            }
                        }
                    }
                }
            ;
        \end{tikzpicture}
        \caption{Suchgraph für die Top-Down Ableitung}
        \label{fig:2}
    \end{figure}
    %\vspace*{\fill}
\end{landscape}


\end{document}
