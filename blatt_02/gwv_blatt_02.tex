\documentclass[a4paper]{scrartcl}

% font/encoding packages
\usepackage[utf8]{inputenc}
\usepackage[T1]{fontenc}
\usepackage{lmodern}
\usepackage[ngerman]{babel}
\usepackage[ngerman=ngerman-x-latest]{hyphsubst}

\usepackage{amsmath, amssymb, amsfonts, amsthm}
\usepackage{array}
\usepackage{stmaryrd}
\usepackage{marvosym}
\allowdisplaybreaks
\usepackage[output-decimal-marker={,}]{siunitx}
\usepackage[shortlabels]{enumitem}
\usepackage[section]{placeins}
\usepackage{float}
\usepackage{units}
\usepackage{listings}
\usepackage{pgfplots}
\pgfplotsset{compat=1.12}
\usepackage[hyphens]{url}
\usepackage{hyperref}

\newtheorem*{behaupt}{Behauptung}
\newcommand{\gdw}{\Leftrightarrow}

\usepackage{fancyhdr}
\pagestyle{fancy}

\def \blattnr {2}

\lhead{GWV - Blatt {\blattnr}}
\rhead{Billis, Braun, Knapperzbusch, Nikolaisen}
\cfoot{\thepage}


\title{Grundlagen der Wissensverarbeitung}
\subtitle{Blatt {\blattnr} Hausaufgaben}
\author{
    Fabian Billis (6720351) \\
    Lennart Braun (6523742), \\
    Maximilian Knapperzbusch (6535090) \\
    Laurens Nikolaisen (TODO) \\
    Foo Bar (TODO)
}
\date{zum 26. Oktober 2015}

\begin{document}
\maketitle

\section*{Exercise \blattnr.1: Search Space 1}
\begin{enumerate}
    \item
        Wir würden den Zustandsraum als Graphen $G = (V,E)$ modellieren.
        Seien $H$ die Menge der Haltestellen und $T$ eine diskrete Menge von
        Zeiteinheiten, die den zu betrachtenden Zeitraum beschreibt (z.~B.
        Minuten ab Beginn des Betriebs). Weiterhin enthalte $L$ alle Bus- und
        Bahnlinien.
        Die Menge unserer Zustände sei dann $V = H \times T$ und enthält damit
        jede Haltestelle zu jedem Zeitpunkt.
        Die Kanten $E \subseteq V \times V \times L$ repräsentieren die
        vorhandenen Verbindungen zwischen den Haltestellen mit Beachtung der
        Fahrzeit.

        Beispiel:
        Betrachen wir die Linie 281, die zum Zeitpunkt $t_0$ an der Haltestelle
        Informatikum abfährt, um $t_1$ am Rathaus Stellingen hält und
        Hagenbecks Tierpark um $t_2$ erreicht. Dann enthält $E$ die Kanten
        \begin{gather*}
            (Informatikum, t_0) \overset{281}{\to} (Rathaus\_Stellingen, t_1) \\
        \text{und} \\
        (Rathaus\_Stellingen, t_1) \overset{281}{\to} (Hagenbecks\_Tierpark, t_2)
        \text{ .}
        \end{gather*}
        In diesem Fall wird die Zeit, in der der Bus an der Haltestelle steht
        nicht gesondert modelliert.

        Es kann bzw. muss natürlich auch gewartet werden, d.~h. $(h,t)
        \overset{\lambda}{\to} (h, u)$ für alle Haltestellen $h \in H$ und
        Zeitpunkte $t,u \in T$ mit $t < u$.

    \item
        \begin{enumerate}[label=(\alph*)]
            \item
                Die Menge der Zustände sei
                \begin{equation*}
                    V = \{0,1,2,3,4\} \times \{0,1,2,3\} \text{ ,}
                \end{equation*}
                wobei $(a,b) \in V$ bedeutet, dass der größere Krug $a$ Liter
                und der kleinere $b$ Liter enthält.
                Eine explizite Auflistung aller Zustände:
                \begin{lstlisting}
In [1]: print('V = {\n' +
                    ',\n'.join(str((a,b)) for a in range(5)
                                          for b in range(4)) +
                    '\n}')
V = {
    (0, 0),
    (0, 1),
    (0, 2),
    (0, 3),
    (1, 0),
    (1, 1),
    (1, 2),
    (1, 3),
    (2, 0),
    (2, 1),
    (2, 2),
    (2, 3),
    (3, 0),
    (3, 1),
    (3, 2),
    (3, 3),
    (4, 0),
    (4, 1),
    (4, 2),
    (4, 3)
}

                \end{lstlisting}

                Zu Beginn sind beide Krüge leer, der Startzustand ist also
                $(0,0)$. Das Ziel ist, im größeren Krug 2 Liter Wasser zu
                haben.  Die Menge der Zielzustände ist also $F = \{(a,b) \in V
                \ |\ a = 2\}$.


                Die Transitionen definieren wir als (rekursive) Funktionen.
                \begin{itemize}
                    \item Einen Krug aus der Pumpe befüllen:
                        \begin{align*}
                            pa(a,b) &= (4,b) \\
                            pb(a,b) &= (a,3) \\
                        \end{align*}
                    \item Einen Krug in den Abfluss entleeren:
                        \begin{align*}
                            ad(a,b) &= (0,b) \\
                            bd(a,b) &= (a,0) \\
                        \end{align*}
                    \item Den Inhalt des einen Kruges soweit wie möglich in den
                        zweiten überführen:
                        \begin{align*}
                            ab(a,b) &=
                            \begin{cases}
                                (a,b) & \text{if } a = 0 \lor b = 3 \\
                                ab(a-1,b+1) & \text{else}
                            \end{cases} \\
                            ba(a,b) &=
                            \begin{cases}
                                (a,b) & \text{if } a = 4 \lor b = 0 \\
                                ab(a+1,b-1) & \text{else}
                            \end{cases} \\
                        \end{align*}
                \end{itemize}

                Eine Lösungsmöglichkeit ist die folgende:
                \begin{equation*}
                    (0,0)
                    \overset{pa}{\to}
                    (4,0)
                    \overset{ab}{\to}
                    (1,3)
                    \overset{bd}{\to}
                    (1,0)
                    \overset{ab}{\to}
                    (0,1)
                    \overset{pa}{\to}
                    (4,1)
                    \overset{ab}{\to}
                    (2,3)
                \end{equation*}

            \item
			Das Rätsel kann nicht gelöst werden, wenn man den Wein nicht wegschütten darf. Es ist jedoch 
			möglich, dass Rätseln mit eingeschränkten Regeln zu lösen, wenn die $2$  gewünschten Liter 
			Wein, in dem $3$ Liter Krug sein dürfen. \\

        \end{enumerate}

\end{enumerate}

\section*{Exercise \blattnr.2: Search Space 2}
\begin{enumerate}
	\item 
	Wir definieren den Zustandsraum als Graph $G = (V, E)$. $V$ beschreibt die Stellung der Spieler auf 
	dem Schachbrett. Der Startzustand beschreibt die Startellung der Figuren und sagt, dass der Spieler
	der weißen Figuren am Zug ist: $Start = \{(Startstellung, weiss)\}$. \\
	Die Endzustände beschreiben welcher Spieler den anderen schachmatt gesetzt hat: $Ende = \{(St, Sp)\}$ 
	beschreibt, dass der König auf der Stellung $Sp$ schachmatt gesetzt wurde. \\
	Die Transitionen beschreiben die einzelnen Züge der Spieler und sehen wie folgt aus: \\
	$(St, Sp) \rightarrow (St', Sp')$, wenn $Sp != Sp'$ ist, dann gibt es einen Zug von $Sp$ von der 
	Stellung $St$ nach $St'$. \\
	
\end{enumerate}


\end{document}
