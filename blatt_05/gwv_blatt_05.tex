\documentclass[a4paper]{scrartcl}

% font/encoding packages
\usepackage[utf8]{inputenc}
\usepackage[T1]{fontenc}
\usepackage{lmodern}
\usepackage[ngerman]{babel}
\usepackage[ngerman=ngerman-x-latest]{hyphsubst}

\usepackage{amsmath, amssymb, amsfonts, amsthm}
\usepackage{array}
\usepackage{stmaryrd}
\usepackage{marvosym}
\allowdisplaybreaks
\usepackage[output-decimal-marker={,}]{siunitx}
\usepackage[shortlabels]{enumitem}
\usepackage[section]{placeins}
\usepackage{float}
\usepackage{units}
\usepackage{listings}
\usepackage{pgfplots}
\pgfplotsset{compat=1.12}
\usepackage[hyphens]{url}
\usepackage{hyperref}

\usepackage{tikz}
\usetikzlibrary{arrows,automata}
\usepackage{verbatim}

%\usepackage[newfloat]{minted}

\lstset{
    language=Python,
    numbers=left,
    frame=single,
    basicstyle=\footnotesize\ttfamily,
    otherkeywords={with,as},
}

\newtheorem*{behaupt}{Behauptung}
\newcommand{\gdw}{\Leftrightarrow}

\usepackage{fancyhdr}
\pagestyle{fancy}

\def \blattnr {5}

\lhead{GWV - Blatt {\blattnr}}
\rhead{Billis, Braun, Knapperzbusch, Nikolaisen}
\cfoot{\thepage}


\title{Grundlagen der Wissensverarbeitung}
\subtitle{Blatt {\blattnr} Hausaufgaben}
\author{
    Fabian Billis (6720351) \\
    Lennart Braun (6523742), \\
    Maximilian Knapperzbusch (6535090) \\
    Laurens Nikolaisen (6527179) \\
}
\date{zum 16. November 2015}

\begin{document}
\maketitle

\section*{Exercise \blattnr.2: Search and Parsing}

\begin{enumerate}
    \item
        \begin{enumerate}[label=(\alph*)]
            \item
            
            {
            
            \textbf{left-arc}   $\langle n|S,n'|I,A\rangle \rightarrow \langle S,n'|I,A \cup \{(n',n)\}\rangle$
            
            Es wird eine Kante vom Nachfolgeknoten n' zum Knoten n erstellt. n befindet sich oben auf dem Stack.
            Anschließend wird n vom Stack entfernt. Dies kann nur durchgeführt werden, wenn die Grammatik die entsprechende Regel LEX(n) $\leftarrow$ LEX(n') enthält und es keine weitere Kante zum Knoten n gibt. 
            
            \textbf{right-arc} $\langle n |S, n' | I, A \rangle \rightarrow \langle n'|n|S, I, A \cup \{(n, n')\}\rangle$ 
            
            
            Hier wird eine Kante vom Knoten n, der sich oben auf dem Stack befindet, zum nächsten Eingabeknoten n' erstellt.
            Dies muss ebenfalls durch eine entsprechende Regel der Grammatik erlaubt sein. Der Knoten n' wird anschließend oben auf dem Stack abgelegt.
            
            \textbf{reduce} $\langle n|S,I,A\rangle \rightarrow \langle S,I,A \rangle$
            
            Der Knoten n, welcher sich oben auf dem Stack befindet, wird von diesem entfernt. Diese Operation wird dann benötigt, wenn ein Knoten mehrere abhängige Knoten "`nach rechts"' besitzt. In diesem Fall müssen zuerst die Knoten vom Stack entfernt werden, die näher an n liegen, bevor die Knoten, die weiter entfernt sind, hinzugefügt werden können. Dabei muss der Knoten, der entfernt wird, einen Kopf besitzen.
            
            \textbf{shift} $\langle S,n|I,A\rangle \rightarrow \langle n|S,I,A \rangle$
            
            Bei dieser Operation wird der nächste Eingabeknoten auf dem Stack abgelegt. Shift wird benötigt, wenn ein Knoten mehrere abhängige Knoten "`nach links"' hat. Die entsprechenden Knoten müssen zuerst durch \textbf{reduce} vom Stack entfernt werden, bevor sie in der richtigen Reihenfolge wieder hinzugefügt werden. Für diese Operation gibt es nur die Bedingung, dass die Eingabeliste nicht leer ist.
        }
			
            \item
            {
            \textbf{termination} $\langle S, \textbf{nil}, A \rangle$
            
            Der Algorithmus terminiert, sobald er eine Konfiguration erreicht, welche $\langle S, \textbf{nil}, A \rangle$ entspricht. Wir befinden uns nun in einem Zustand, in welchem die Eingabe zu Ende gelesen wurde. Dies bedeutet, dass der der Eingabe-String $W$ "`geparsed"' \ wurde, somit der der Input-Token der leeren Menge \textbf{nil} entspricht. Ob Daten auf dem Stack liegen ist unerheblich.
	        }

            \item
            {            
            \textbf{directed graph} $D = (N_W, A)$
            nodes $N_W$ set of tokens in $W$
            
            Jeder Knoten ist ein Symbol/Wort/Satzbaustein des zu analysierenden Satzes.
            
            \textbf{Wohlgeformtheit}: \\
            \begin{itemize}
           
            \item {\textbf{Single head}  $(\forall n n' n'') (n \rightarrow n' \land n'' \rightarrow n') \Rightarrow n = n''$
            
            Jeder Knoten hat genau einen Vaterknoten.
	        }
            
            \item {\textbf{Acyclic}      $(\forall n n') \lnot (n \rightarrow n' \land n' \rightarrow^\ast n)$
            
            Der Graph enthält keine Zyklen, wodurch der komplette Graph gelesen werden kann, ohne dass eine Suche in einer Schleife landet und andere Knoten nicht mehr beachtet bzw. gefunden/gelesen werden. Durch diese Eigenschaft ist auch Irreflexivität gegeben, da kein Knoten eine Kante zu sich selbst haben darf, nach der Regel.
	        }
            
            \item {\textbf{Connected}    $(\forall n n') n \leftrightarrow^\ast n'$
            
            Es existiert ein Pfad von jedem Knoten zu jedem Knoten in den Graph.
	        }
            
            \item  {\textbf{Projective} $(\forall n n' n'') (n \leftrightarrow n') \land n < n'' < n') \rightarrow (n \rightarrow^\ast n'' \lor n' \rightarrow^\ast n'')$
	        Wenn es in dem Graphen einen Pfad von $n$ zu $n'$ existiert, dann muss es auch für alle
			$n''$, für die gilt: $n < n'' < n'$ einen Pfad von $n$ oder $n'$ zu $n''$ geben.
	        }
		    \end{itemize}
	        }
            
            \item
            Als Vergleich sei zunächst ein korrekter Graph gegeben. Dier erfüllt die folgenden Bedingungen.
            
            \begin{itemize}
            	\item Single Head
            	\item Acyclic
            	\item Connected
            	\item Projective
            \end{itemize}
            
\begin{figure}
	\centering

    \begin{tikzpicture}[%
        ->,
        >=stealth',
        scale=2,
        semithick,
    ]

        \node[circle,draw] (A) at (0,1) {$A$};
        \node[circle,draw] (B) at (1,2) {$B$};
        \node[circle,draw] (C) at (2,1) {$C$};
        \node[circle,draw] (D) at (3,3) {$D$};
        \node[circle,draw] (E) at (4,2) {$E$};

        \draw (B) -> (A);
        \draw (D) -> (B);
        \draw (B) -> (C);
        \draw (D) -> (E);

    \end{tikzpicture}
	
	\caption{wohlgeformter Graph}
	\label{fig:P01}
\end{figure}

\begin{figure}
	\centering
	
    \begin{tikzpicture}[%
        ->,
        >=stealth',
        scale=2,
        semithick,
    ]

        \node[circle,draw] (A) at (0,1) {$A$};
        \node[circle,draw] (B) at (1,2) {$B$};
        \node[circle,draw] (C) at (2,1) {$C$};
        \node[circle,draw] (D) at (3,3) {$D$};
        \node[circle,draw] (E) at (4,2) {$E$};

        \draw (B) -> (A);
        \draw (D) -> (B);
        \draw (B) -> (C);
        \draw (D) -> (E);
        \draw (E) -> (C);

    \end{tikzpicture}
	
	\caption{"`Single Head"'-Bedingung verletzt}
	\label{fig:P02}
\end{figure}

\begin{figure}
	\centering
	\begin{tikzpicture}[%
        ->,
        >=stealth',
        scale=2,
        semithick,
    ]

        \node[circle,draw] (A) at (0,1) {$A$};
        \node[circle,draw] (B) at (1,2) {$B$};
        \node[circle,draw] (C) at (2,1) {$C$};
        \node[circle,draw] (D) at (3,3) {$D$};
        \node[circle,draw] (E) at (4,2) {$E$};

        \draw (B) -> (A);
        \draw (D) -> (B);
        \draw (B) -> (C);
        \draw (D) -> (E);
        \draw (A) -> (C);

    \end{tikzpicture}
	
	\caption{"`Acyclic"'-Bedingung verletzt}
	\label{fig:P03}
\end{figure}

\begin{figure}
	\centering
	
	\begin{tikzpicture}[%
        ->,
        >=stealth',
        scale=2,
        semithick,
    ]

        \node[circle,draw] (A) at (0,1) {$A$};
        \node[circle,draw] (B) at (1,2) {$B$};
        \node[circle,draw] (C) at (2,1) {$C$};
        \node[circle,draw] (D) at (3,3) {$D$};
        \node[circle,draw] (E) at (4,2) {$E$};

        \draw (B) -> (A);
        \draw (D) -> (B);
        \draw (D) -> (E);

    \end{tikzpicture}
	
	\caption{"`Connected"'-Bedingung verletzt}
	\label{fig:P04}
\end{figure}


\begin{figure}
	\centering
	
	\begin{tikzpicture}[%
        ->,
        >=stealth',
        scale=2,
        semithick,
    ]

        \node[circle,draw] (A) at (0,1) {$A$};
        \node[circle,draw] (B) at (1,2) {$B$};
        \node[circle,draw] (C) at (2,2) {$C$};
        \node[circle,draw] (D) at (3,3) {$D$};
        \node[circle,draw] (E) at (4,2) {$E$};

        \draw (C) -> (A);
        \draw (D) -> (B);
        \draw (D) -> (C);
        \draw (D) -> (E);

    \end{tikzpicture}
	
	\caption{"`Projective"'-Bedingung verletzt}
	\label{fig:P05}
\end{figure}

        \end{enumerate}
		
        \FloatBarrier
		
    \item
        \begin{align*}
            \emph{d} &= Der \\
            \emph{m} &= Mann \\
            \emph{i} &= isst \\
            \emph{e} &= eine \\
            \emph{g} &= Giraffe \\
            W &= Der\ Mann\ isst\ eine\ Giraffe  = dmieg \\
            I &= [(0, Der), (1, Mann), (2, isst), (3, eine), (4, Giraffe)] \\
            N_W &= \{(0, Der), (1, Mann), (2, isst), (3, eine), (4, Giraffe)\}
        \end{align*}
        \begin{gather*}
            \Big\langle \mathbf{nil},\ 
            \big[ (0, d), (1, m), (2, i), (3, e), (4, g) \big],\ 
            \emptyset \Big\rangle \\
            \xrightarrow{\text{Shift}} \\
            \Big\langle \big[ (0, d) \big],\ 
            \big[ (1, m), (2, i), (3, e), (4, g) \big],\ 
            \emptyset \Big\rangle \\
            \xrightarrow{\text{Left-Arc}} \\
            \Big\langle \mathbf{nil},\ 
            \big[ (1, m), (2, i), (3, e), (4, g) \big],\ 
            \big\{ \langle(1, m), (0, d) \rangle
            \big\} \Big\rangle \\
            \xrightarrow{\text{Shift}} \\
            \Big\langle \big[ (1, m) \big],\ 
            \big[ (2, i), (3, e), (4, g) \big],\ 
            \big\{ \langle (1, m), (0, d) \rangle
            \big\} \Big\rangle \\
            \xrightarrow{\text{Left-Arc}} \\
            \Big\langle \mathbf{nil},\
            \big[ (2, i), (3, e), (4, g) \big],\ 
            \big\{ \langle (1, m), (0, d) \rangle,
                   \langle (2, i), (1, m) \rangle
            \big\} \Big\rangle \\
            \xrightarrow{\text{Shift}} \\
            \Big\langle \big[ (2, i) \big],\ 
            \big[ (3, e), (4, g) \big],\ 
            \big\{ \langle (1, m), (0, d) \rangle, 
                   \langle (2, i), (1, m) \rangle
            \big\} \Big\rangle \\
            \xrightarrow{\text{Shift}} \\
            \Big\langle \big[ (3, e), (2, i) \big],\ 
            \big[ (4, g) \big],\ 
            \big\{ \langle (1, m), (0, d) \rangle,
                   \langle (2, i), (1, m) \rangle
            \big \} \Big \rangle \\
            \xrightarrow{\text{Left-Arc}} \\
            \Big\langle \big[ (2, i) \big],\ 
            \big[ (4, g) \big],\ 
            \big\{ \langle (1, m), (0, d) \rangle,
                   \langle (2, i), (1, m) \rangle, 
                   \langle (4, g), (3, e) \rangle
            \big\} \Big\rangle \\
            \xrightarrow{\text{Right-Arc}} \\
            \Big\langle \big[ (4, g), (2, i) \big],\
            \mathbf{nil},\ 
            \big\{ \langle (1, m), (0, d) \rangle,
                   \langle (2, i), (1, m) \rangle,
                   \langle (4, g), (3, e) \rangle,
                   \langle (2, i), (4, g) \rangle
            \big\} \Big\rangle \\
        \end{gather*}
\newpage
    \item
		\begin{enumerate}
			\item
				Für die Suchzustände gilt: $s = \langle S,I,A \rangle$ \\
				Wobei $S$ für den Stack steht, $I$ den Input umschreibt und $A$ eine Liste der Relationen 
				ist.
				
			\item
				Der Startzustand wird durch einen leeren Stack $S$, einer mit dem Input gefühlten Liste
				$I$ und einer leeren Liste $A$, da noch keine Relationen gefunden wurden. Der Starzustand 
				$start$ ist demnach, wie folgt deffiniert: \\ 
				$start = \langle \emptyset,I,\emptyset \rangle$.
				
			\item
				Ein Endzustand ist erreicht, wenn die Liste mit einem Input $I$ leer ist, also der Input 
				vollständig eingelsen wurde und die Ausgabe akzeptiert wurde.
				Daher gilt für einen Endzustand: \\ 
				$ende = \langle S,\emptyset,A \rangle$
				
			\item
				Die Operationen des Parsers dienen als Kantenübergang zwischen den einzelnen Zuständen.
				Des Weiteren existieren die Kanten nur dann wenn die entsprechenden Bedinungen erfüllt 
				sind. \\
				
		       \textbf{left-arc}   $\langle n|S,n'|I,A\rangle \rightarrow \langle S,n'|I,A \cup \{(n',n)\}\rangle$ \\
	            \textbf{right-arc} $\langle n |S, n' | I, A \rangle \rightarrow \langle n'|n|S, I, A \cup \{(n, n')\}\rangle$ \\
	            \textbf{reduce} $\langle n|S,I,A\rangle \rightarrow \langle S,I,A \rangle$ \\
	            \textbf{shift} $\langle S,n|I,A\rangle \rightarrow \langle n|S,I,A \rangle$ \\
				
			\item
				Der Suchraum kann vorher erstellt werden, es macht jedoch keinen Sinn, da dies
				mehr Aufwand erzeugen würde, als die Suche selber. \\
				
			\item
				Der Vorteil des Algorthmuses ist, dass der Algorithmus in linearer Zeit terminiert und der
				Aufwand, dadurch geringer ist, als alle Möglichkeiten auszuprobieren. So braucht der 
				Algorithmus bei einem Input der Länge $n$ maximal $2n$ Transitionen um zu terminieren.
				
			\item
				Mit der Breiten- und Tiefensuche kann man das Problem recht gut lösen, da hier nur 
				die Vor- und Nachteile der beiden Suchen beachtet werden und geguckt werden muss welche 
				Suche besser geeignet ist. Beide Suchen werden terminieren, da der Input endlich ist und
				der Parser terminieren wird, d.h. bei der Anwendung der Tiefensuche kann es nicht 
				passieren, dass ein Pfad immer tiefer führt, ohne zu terminieren. Jedoch kann es 
				sein, dass man eine Laufzeit von $n^2$ haben kann, da man im schlimmsten Fall alle
				Pfade überprüfen muss. Bei der Breitensuche ist der Vorteil, dass man von jedem Zustand
				aus alle Operationen betrachten kann. Des Weiteren kann der Aufwand der Breistensuche
				geringer sein als bei der Tiefensuche. \\
				Bei der $A^*$-Suche besteht das Problem darin, dass es sehr schwer ist eine passende
				Heuristik zu finden und ohne eine passende Heuristik ist es sehr schwer den Algorithmus
				anzuwenden. \\
				
			\item
				Wir würden den Parser in Kombination mit einer Breitensuche anwenden, um auf diese Weise
				nach jeder Transition zu prüfen welcheer Pfad möglich ist. Auf diese Weise kann man den 
				Aufwand des Parsers möglichst gering halten. Wenn man eine Tiefensuche verwenden würde,
				müsste man hoffen, dass man immer am Anfang einen Pfad findet der zum Ziel führt und bei 
				der Breitensuche kann man in der Regel den Aufwand geringer halten, als bei der 
				Tiefensuche.
				
				
				
				
		\end{enumerate}

\end{enumerate}


\end{document}
