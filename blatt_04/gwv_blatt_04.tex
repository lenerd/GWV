\documentclass[a4paper]{scrartcl}

% font/encoding packages
\usepackage[utf8]{inputenc}
\usepackage[T1]{fontenc}
\usepackage{lmodern}
\usepackage[ngerman]{babel}
\usepackage[ngerman=ngerman-x-latest]{hyphsubst}

\usepackage{amsmath, amssymb, amsfonts, amsthm}
\usepackage{array}
\usepackage{stmaryrd}
\usepackage{marvosym}
\allowdisplaybreaks
\usepackage[output-decimal-marker={,}]{siunitx}
\usepackage[shortlabels]{enumitem}
\usepackage[section]{placeins}
\usepackage{float}
\usepackage{units}
\usepackage{listings}
\usepackage{pgfplots}
\pgfplotsset{compat=1.12}
\usepackage[hyphens]{url}
\usepackage{hyperref}
%\usepackage[newfloat]{minted}

\lstset{
    language=Python,
    numbers=left,
    frame=single,
    basicstyle=\footnotesize\ttfamily,
    otherkeywords={with,as},
}

\newtheorem*{behaupt}{Behauptung}
\newcommand{\gdw}{\Leftrightarrow}

\usepackage{fancyhdr}
\pagestyle{fancy}

\def \blattnr {4}

\lhead{GWV - Blatt {\blattnr}}
\rhead{Billis, Braun, Knapperzbusch, Nikolaisen, Bar}
\cfoot{\thepage}


\title{Grundlagen der Wissensverarbeitung}
\subtitle{Blatt {\blattnr} Hausaufgaben}
\author{
    Fabian Billis (6720351) \\
    Lennart Braun (6523742), \\
    Maximilian Knapperzbusch (6535090) \\
    Laurens Nikolaisen (6527179) \\
    Foo Bar (TODO)
}
\date{zum 9. November 2015}

\begin{document}
\maketitle

\section*{Exercise \blattnr.2: Heuristic Search}

\begin{enumerate}
    \item
        Als Heuristik haben wir die Manhattan Distanz gewählt. Diese entspräche
        der Länge der kürzesten Pfade, wenn es keine Hindernisse gäbe.
        Daher ist die Manhattan Distanz eine optimistische Heuristik und eine
        untere Schranke für die Länge des Pfades.

    \item
        Da unsere Implementation jeden schon besuchten Knoten markiert,
        terminiert die Suche, sobald jeder der endlich vielen Knoten besucht
        wurde.

    \item

    \item

\end{enumerate}


\end{document}
