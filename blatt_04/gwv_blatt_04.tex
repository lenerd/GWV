\documentclass[a4paper]{scrartcl}

% font/encoding packages
\usepackage[utf8]{inputenc}
\usepackage[T1]{fontenc}
\usepackage{lmodern}
\usepackage[ngerman]{babel}
\usepackage[ngerman=ngerman-x-latest]{hyphsubst}

\usepackage{amsmath, amssymb, amsfonts, amsthm}
\usepackage{array}
\usepackage{stmaryrd}
\usepackage{marvosym}
\allowdisplaybreaks
\usepackage[output-decimal-marker={,}]{siunitx}
\usepackage[shortlabels]{enumitem}
\usepackage[section]{placeins}
\usepackage{float}
\usepackage{units}
\usepackage{listings}
\usepackage{pgfplots}
\pgfplotsset{compat=1.12}
\usepackage[hyphens]{url}
\usepackage{hyperref}
%\usepackage[newfloat]{minted}

\lstset{
    language=Python,
    numbers=left,
    frame=single,
    basicstyle=\footnotesize\ttfamily,
    otherkeywords={with,as},
}

\newtheorem*{behaupt}{Behauptung}
\newcommand{\gdw}{\Leftrightarrow}

\usepackage{fancyhdr}
\pagestyle{fancy}

\def \blattnr {4}

\lhead{GWV - Blatt {\blattnr}}
\rhead{Billis, Braun, Knapperzbusch, Nikolaisen}
\cfoot{\thepage}


\title{Grundlagen der Wissensverarbeitung}
\subtitle{Blatt {\blattnr} Hausaufgaben}
\author{
    Fabian Billis (6720351) \\
    Lennart Braun (6523742), \\
    Maximilian Knapperzbusch (6535090) \\
    Laurens Nikolaisen (6527179) \\
}
\date{zum 9. November 2015}

\begin{document}
\maketitle

\section*{Exercise \blattnr.2: Heuristic Search}

\begin{enumerate}
    \item
        Als Heuristik haben wir die Manhattan-Distanz gewählt. Diese entspräche
        der Länge der kürzesten Pfade, wenn es keine Hindernisse gäbe.
        Daher ist die Manhattan-Distanz eine optimistische Heuristik und eine
        untere Schranke für die Länge des Pfades.

    \item
        Da unsere Implementation jeden schon besuchten Knoten markiert,
        terminiert die Suche, sobald jeder der endlich vielen Knoten besucht
        wurde.

    \item
        Befindet sich ein Knoten neben einem Portal, so wird der Ausgangsknoten
        des Portals als Nachbar bezeichnet.
        Der kürzeste Pfad zum Ziel könnte jedes Paar von Portalen einmal in
        einer Richtung benutzen oder auch nicht. Die mehrmalige Benutzung von
        Portalen würde einen Kreis darstellen und ist damit überflüssig.

        Durch die Portale ist nicht mehr gesichert, dass die Manhattan-Distanz
        eine optimistische Abschätzung ist.
        Daher berechnen wir zunächst Abschätzungen für jeden Portaleingang.
        Diese entspricht der minimalen Summe der Manhattan-Distanzen zwischen
        den Portalen und der anschließenden Manhatten-Distanz zum Ziel auf
        allen Routen unter möglicher Verwendung der die existierenden Portale.
        Die Anzahl der Routen steigt exponentiell mit der Anzahl der Portale.
        Daher ist diese Vorgehensweise nur bei wenigen Portalen zu empfehlen.

        Während der A*-Suche wird als Heuristic für einen Knoten das Minimum
        der Manhatten-Distanz zum Ziel und den Manhattan-Distanzen zu allen
        Portaleingängen verwendet.
        

    \item
        Tabelle \ref{tab:time-space} zeigt eine Übersicht, wie sich die drei
        implementierten Algorithmen in den drei Labyrinthen geschlagen haben.
        Die Auswahl der Nachbarknoten folgt einer willkürlich gewählten
        Reihenfolge.
        \begin{table}[H]
            \centering
            \begin{tabular}{|l|r|r|r|}
                \hline
                Algorithm & Time & Space & Length \\ \hline
                \multicolumn{4}{|l|}{\texttt{blatt3\_environment.txt}} \\ \hline
                A*  &  85 &  39 & 25 \\ \hline
                BFS &  92 & 124 & 25 \\ \hline
                DFS &  62 & 653 & 49 \\ \hline
                \multicolumn{4}{|l|}{\texttt{blatt4\_environment\_a.txt}} \\ \hline
                A*  &  99 &  39 & $\infty$ \\ \hline
                BFS &  99 &  94 & $\infty$ \\ \hline
                DFS &  99 & 614 & $\infty$ \\ \hline
                \multicolumn{4}{|l|}{\texttt{blatt4\_environment\_b.txt}} \\ \hline
                A*  &  21 &  51 & 11 \\ \hline
                BFS &  79 & 121 & 11 \\ \hline
                DFS &  19 & 144 & 37 \\ \hline
            \end{tabular}
            \caption{Time als Anzahl der Iterationen; Space als Anzahl der
            Knoten in allen Pfaden der Frontier}
            \label{tab:time-space}
        \end{table}
i
\end{enumerate}


\end{document}
