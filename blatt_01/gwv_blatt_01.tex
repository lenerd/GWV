\documentclass[a4paper]{scrartcl}

% font/encoding packages
\usepackage[utf8]{inputenc}
\usepackage[T1]{fontenc}
\usepackage{lmodern}
\usepackage[ngerman]{babel}
\usepackage[ngerman=ngerman-x-latest]{hyphsubst}

\usepackage{amsmath, amssymb, amsfonts, amsthm}
\usepackage{array}
\usepackage{stmaryrd}
\usepackage{marvosym}
\allowdisplaybreaks
\usepackage[output-decimal-marker={,}]{siunitx}
\usepackage[shortlabels]{enumitem}
\usepackage[section]{placeins}
\usepackage{float}
\usepackage{units}
\usepackage{listings}
\usepackage{pgfplots}
\pgfplotsset{compat=1.12}

\newtheorem*{behaupt}{Behauptung}
\newcommand{\gdw}{\Leftrightarrow}

\usepackage{fancyhdr}
\pagestyle{fancy}

\def \blattnr {1 }

\lhead{GWV - Blatt \blattnr}
\rhead{Billis, Braun, Knapperzbusch, Nikolaisen}
\cfoot{\thepage}


\title{Grundlagen der Wissensverarbeitung}
\subtitle{Blatt \blattnr Hausaufgaben}
\author{
    Fabian Billis (TODO) \\
    Lennart Braun (6523742), \\
    Maximilian Knapperzbusch (6535090) \\
    Laurens Nikolaisen (TODO)
}
\date{zum 19. Oktober 2015}

\begin{document}
\maketitle

\section*{Exercise 1.1: Application Scenarios for Artificial Intelligence}
  \begin{enumerate}
    \item{Schach} \\
      Die Aufgabe der AI besteht darin den Gegner auszuspielen und die Partie zu
      gewinnen. Für die AI gegeben sind die Spielfiguren mit ihren jeweiligen
      Eigenschaften, sowie ihrer Position auf dem Spielfeld. Des Weiteren ist der
      Aufbau des Spielbretts bekannt.
      Die Aufgabe ist dann erfüllt, wenn die AI ihren Gegenspieler schachmatt gesetzt
      hat. \\

      Warum wird Intelligenz benötigt? \\
      Für die Bewältigung dieser Aufgabe wird Intelligenz benötigt, da die aktuelle
      Lage auf dem Spielfeld eingeschätzt werden muss und aufgrund dieser Einschätzung
      eine Entscheidung über den nächsten Zug getroffen werden muss. Da sich die Lage
      jederzeit auf dem Spielfeld ändern, muss die AI flexibel auf Handlungen
      des Gegenspielers reagieren. \\

      Was für Schwierigkeiten teten auf?\\
      Schwierigkeiten können bei der Planung einer Abfolge von Spielzügen auftreten,
      da mögliche Handlungen des Gegenspielers eingeplant bzw. erahnt werden müssen.
      Außerdem kann es Probleme geben, der letzte Zug des Gegners die voraus geplanten
      Züge zunichte macht, in diesem Fall muss eine neue Strategie innerhabl kürzester
      Zeit entwickelt werden. \\

      Link: http://www.shredderchess.de/online-schach-spielen.html \\



    \item{Wegfindung}
      Die AI soll den optimalen oder gewünschten Weg von einem Ort zum anderen finden.
      Dabei muss unterschieden werden, ob der schnellste Weg genommen werden soll, oder
      ob bestimmte Straßentypen nicht genutzt werden sollen, z.B. Bundesstraßen oder
      Autobahn. Nebenbei soll das Navi, sich häufig genutzte Routen merken und diese Überwachen, um
      die optimale Route zu finden. Um die Aufgabe zu lösen stehen Kartenmarterial, eine GPS-Verbindung
      sowie Informationen über die aktuelle Verkehrslage zur Verfügung.  \\

      Warum wird Intelligenz benötigt? \\
      Die AI muss in der Lage sein, die Verkehrslage auf der geplanten Route zu
      überwachen und notfalls eine neue Route berechnen, um mögliche Staus oder
      Baustellen zu reagieren. Des Weiteren  muss das System in der Lage sein Muster zu erkenn
      und diese zu optimieren und zu verbessern.\\

      Was für Schwierigkeiten treten auf? \\
      Es können Konflikte zur Laufzeit auftreten, wenn das System einen Stau umgehen
      möchte, die alternative Route aber in Konflikt mit den Wünschen des Nutzers steht,
      muss die AI eine Entscheidung treffen, die zum optimalen Weg fürht. \\

      Link: http://www.cnet.com/news/bmw-developing-artificially-intelligent-navigation/ \\

    \item{Gesichtserkennung}
      Die AI soll in der Lage sein Gesichter auf eingelesenen/hochgeladenen Bildern zu erkennen und zu
      verarbeiten. Neben der einfachen Erkennung der Gesichter soll das System in der Lage sein Gesichter
      wieder zu erkennen und diesen zum Beispiel einen Namen oder eine Akte zu zuweisen. Als Eingabewerte
      dienen Bilder. \\

      Warum wird Intelligenz benötigt? \\
      Für die Erfüllung dieser Aufgabe wird Intelligenz benötigt, da das eingelesene
      Bild im Kontext der vorhandenen Informationen interpretiert werden muss. \\

      Was für Schwierigkeiten treten auf? \\
      Es können verschiedene Probleme bei der Gesichtserkennung geben, da bei den Bildern
      der Winkel und die Beleuchtung des Bildes die Qualität des Gesichtes verschlechtern
      können und so eine Erkennung erschweren. Außerdem muss beachtet werden, dass es keine
      Normen, wie beim Personalausweis, gibt, da das System auch an Flughäfen oder anderen
      öffentlichen Orten genutzt werden kann oder in sozialen Netzwerken zum Einsatz kommt. \\

      Link: https://www.facebook.com/ \\

\end{enumerate}


\section*{Exercise 1.2: AI Terminology}

\subsection*{Knowledge}
\begin{itemize}
    \item
        Als \emph{Information} bezeichnet man die Bedeutung einer Anordnung von
        Symbolen. In den Bits auf der Festplatte eines Notebooks ist
        Information kodiert.

    \item
        \emph{Implizites Wissen} beschreibt Wissen, das jemand besitzt, ohne
        dass es formal beschrieben und weitergegeben werden kann. Beispiele
        sind die Fähigkeit eines Menschen, Tätigkeiten intuitiv auszuführen,
        ohne dass der Person bewusst ist wie genau diese funktionieren.

    \item
        Unter \emph{explizitem Wissen} wird im Gegensatz zum impliziten Wissen
        eindeutig kodierbares Wissen verstanden.  Das ganz zum Beispiel eine
        Definition in einem Paper sein.

\end{itemize}

\subsection*{Characteristics of Environments}

\begin{itemize}
    \item
        Ist die Umgebung nur \emph{teilweise beobachtbar}, so muss ggf. auf Dinge im
        momentan nicht beobachtbaren Teil geschlossen werden.

        Beispiel: Ein Roboter soll sich in einem Wegenetz von einem Startpunkt zu einem
        Enpunkt bewegen und dabei eine möglichst kurze Strecke zurücklegen.  Das
        Wegenetz sei als Graph modelliert.  Ist dieser \emph{komplett} bekannt, kann
        z.~B. Dijkstras Algorithmus angewandt werden, um den kürzesten Weg zu finden.
        Andernfalls muss der Roboter einen Weg durch das Labyrinth suchen.

    \item
        \emph{Kontinuierliche} Werte lassen sich von Digitalcomputern nicht verarbeiten,
        daher ist es oft notwendig, Werte zu \emph{diskretisieren}.

        Beispiel: An den Pin eines Mikrocontrollers wird eine Spannung angelegt. Um
        diese digital zu verarbeiten, wird sie mit Hilfe von einem A/D-Wandler in
        diskrete Werte überführt. Dabei gehen Informationen verloren, da ein
        Mikrocontroller Werte nur mit einer endlichen Anzahl an Bits speichern und
        verarbeiten kann.

    \item
        Das Verhalten der Umgebung ist unter Umständen nicht genau bekannt und muss
        daher durch ein \emph{stochastisches} Modell beschrieben werden.

        Beispiel: Es ist nicht bekannt, wann ein Kommunikationspartner kommunizieren
        möchte, aber es kann gesagt werden, dass dies im Mittel einmal alle $t$
        Zeiteinheiten geschieht.
\end{itemize}

\end{document}
