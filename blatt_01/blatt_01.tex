\documentclass[a4paper]{scrartcl}

% font/encoding packages
\usepackage[utf8]{inputenc}
\usepackage[T1]{fontenc}
\usepackage{lmodern}
\usepackage[ngerman]{babel}

% math
\usepackage{amsmath, amssymb, amsfonts, amsthm, mathtools}
\allowdisplaybreaks
\newtheorem*{behaupt}{Behauptung}

% tikz
\usepackage{tikz}
\usetikzlibrary{arrows,automata}

% misc
\usepackage{enumerate}

\title{Grundlagen der Wissenverarbeitung}
\subtitle{Aufgabenblatt 1}
\author{
    Lennart Braun (6523742) \\
    Laurens Nikolaisen () \\
    Fabian Billis () \\
    Maximilian Knapperzbusch (6535090) \\
}
\date{zum 19. Oktober 2015}

\begin{document}
\maketitle

\section{Exercise 1.1}
  \begin{enumerate}
    \item{Schach} \\
      Die Aufgabe der AI besteht darin den Gegner auszuspielen und die Partie zu
      gewinnen. Für die AI gegeben sind die Spielfiguren mit ihren jeweiligen
      Eigenschaften, sowie ihrer Position auf dem Spielfeld. Des Weiteren ist der
      Aufbau des Spielbretts bekannt.
      Die Aufgabe ist dann erfüllt, wenn die AI ihren Gegenspieler schachmatt gesetzt
      hat. \\

      Warum wird Intelligenz benötigt? \\
      Für die Bewältigung dieser Aufgabe wird Intelligenz benötigt, da die aktuelle
      Lage auf dem Spielfeld eingeschätzt werden muss und aufgrund dieser Einschätzung
      eine Entscheidung über den nächsten Zug getroffen werden muss. Da sich die Lage
      jederzeit auf dem Spielfeld ändern, muss die AI flexibel auf Handlungen
      des Gegenspielers reagieren. \\

      Was für Schwierigkeiten teten auf?\\
      Schwierigkeiten können bei der Planung einer Abfolge von Spielzügen auftreten,
      da mögliche Handlungen des Gegenspielers eingeplant bzw. erahnt werden müssen.
      Außerdem kann es Probleme geben, der letzte Zug des Gegners die voraus geplanten
      Züge zunichte macht, in diesem Fall muss eine neue Strategie innerhabl kürzester
      Zeit entwickelt werden. \\

      Link: http://www.shredderchess.de/online-schach-spielen.html



    \item{Wegfindung}
      Die AI soll den optimalen oder gewünschten Weg von einem Ort zum anderen finden.
      Dabei muss unterschieden werden, ob der schnellste Weg genommen werden soll, oder
      ob bestimmte Straßentypen nicht genutzt werden sollen, z.B. Bundesstraßen oder
      Autobahn. Um die Aufgabe zu lösen stehen Kartenmarterial, eine GPS-Verbindung
      sowie Informationen über die aktuelle Verkehrslage zur Verfügung.  \\

      Warum wird Intelligenz benötigt? \\
      Die AI muss in der Lage sein, die 







\end{document}
