\documentclass[a4paper]{scrartcl}

% font/encoding packages
\usepackage[utf8]{inputenc}
\usepackage[T1]{fontenc}
\usepackage{lmodern}
\usepackage[ngerman]{babel}
\usepackage[ngerman=ngerman-x-latest]{hyphsubst}

\usepackage{amsmath, amssymb, amsfonts, amsthm}
\usepackage{array}
\usepackage{stmaryrd}
\usepackage{marvosym}
\allowdisplaybreaks
\usepackage[output-decimal-marker={,}]{siunitx}
\usepackage[shortlabels]{enumitem}
\usepackage[section]{placeins}
\usepackage{float}
\usepackage{units}
\usepackage{listings}
\usepackage{pgfplots}
\pgfplotsset{compat=1.12}
\usepackage[hyphens]{url}
\usepackage{hyperref}

\usepackage{tikz}
\usetikzlibrary{arrows,automata}
\usepackage{verbatim}

%\usepackage[newfloat]{minted}

\lstset{
    language=Python,
    numbers=left,
    frame=single,
    basicstyle=\footnotesize\ttfamily,
    otherkeywords={with,as},
}

\newtheorem*{behaupt}{Behauptung}
\newcommand{\gdw}{\Leftrightarrow}

\usepackage{fancyhdr}
\pagestyle{fancy}

\def \blattnr {6}

\lhead{GWV - Blatt {\blattnr}}
\rhead{Billis, Braun, Knapperzbusch, Nikolaisen}
\cfoot{\thepage}


\title{Grundlagen der Wissensverarbeitung}
\subtitle{Blatt {\blattnr} Hausaufgaben}
\author{
    Fabian Billis (6720351) \\
    Lennart Braun (6523742), \\
    Maximilian Knapperzbusch (6535090) \\
    Laurens Nikolaisen (6527179) \\
}
\date{zum 23. November 2015}

\begin{document}
\maketitle

\section*{Exercise \blattnr.1: Constraints}
\begin{enumerate}
    \item
        Mit Hilfe der Regeln der schriftlichen Addition, kann man das Problem
        formalisieren.
        \begin{center}
            \begin{tabular}{crrrrr}
                  &    & A4 & A3 & A2 & A1 \\
                  &    & B4 & B3 & B2 & B1 \\
                + & C5 & C4 & C3 & C2 & C1 \\ \hline
                  & D5 & D4 & D3 & D2 & D1
            \end{tabular}
        \end{center}
        Für die Domänen gilt:
        \begin{gather*}
            Dom(Ai) = Dom(Bi) = Dom(Di) = \{0,...,9\} \\
            Dom(Ci) = \{0,1\}
        \end{gather*}
        $Ci$ steht für den Übertrag.  Man muss die Domänen der Summanden und
        des Ergebnisses von der des Übertrags trennen, da der Übertrag nur zwei
        verschiedene Werte annehmen kann. Für das Ergebnis der Addition gilt:
        \begin{align*}
            Di &= (Ai + Bi + Ci) \mod 10 \text{ für } i = 1,2,3,4 \\
            D5 &= C5 \\
            C1 &= 0 \\
            C(i+1) &= \lfloor (Ai + Bi + Ci) / 10 \rfloor
            \text{ für }  i = 1,2,3,4
        \end{align*}
	   
    \item

    \item

\end{enumerate}


\end{document}
