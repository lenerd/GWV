\documentclass[a4paper]{scrartcl}

% font/encoding packages
\usepackage[utf8]{inputenc}
\usepackage[T1]{fontenc}
\usepackage{lmodern}
\usepackage[ngerman]{babel}
\usepackage[ngerman=ngerman-x-latest]{hyphsubst}

\usepackage{amsmath, amssymb, amsfonts, amsthm}
\usepackage{array}
\usepackage{stmaryrd}
\usepackage{marvosym}
\allowdisplaybreaks
\usepackage[output-decimal-marker={,}]{siunitx}
\usepackage[shortlabels]{enumitem}
\usepackage[section]{placeins}
\usepackage{float}
\usepackage{units}
\usepackage{listings}
\usepackage{pgfplots}
\pgfplotsset{compat=1.12}
\usepackage[hyphens]{url}
\usepackage{hyperref}
\usepackage{pdflscape}

\usepackage{tikz}
\usetikzlibrary{arrows,automata}
\usepackage{verbatim}

%\usepackage[newfloat]{minted}

\lstset{
    language=Python,
    numbers=left,
    frame=single,
    basicstyle=\footnotesize\ttfamily,
    otherkeywords={with,as},
}

\newtheorem*{behaupt}{Behauptung}
\newcommand{\gdw}{\Leftrightarrow}

\usepackage{fancyhdr}
\pagestyle{fancy}

\def \blattnr {6}

\lhead{GWV - Blatt {\blattnr}}
\rhead{Billis, Braun, Knapperzbusch, Nikolaisen}
\cfoot{\thepage}


\title{Grundlagen der Wissensverarbeitung}
\subtitle{Blatt {\blattnr} Hausaufgaben}
\author{
    Fabian Billis (6720351) \\
    Lennart Braun (6523742), \\
    Maximilian Knapperzbusch (6535090) \\
    Laurens Nikolaisen (6527179) \\
}
\date{zum 23. November 2015}

\begin{document}
\maketitle

\section*{Exercise \blattnr.1: Constraints}
\begin{enumerate}
    \item
        Mit Hilfe der Regeln der schriftlichen Addition, kann man das Problem
        formalisieren.
        \begin{center}
            \begin{tabular}{crrrrr}
                  &    & A4 & A3 & A2 & A1 \\
                  &    & B4 & B3 & B2 & B1 \\
                + & C5 & C4 & C3 & C2 & C1 \\ \hline
                  & D5 & D4 & D3 & D2 & D1
            \end{tabular}
        \end{center}
        Für die Domänen gilt:
        \begin{gather*}
            Dom(Ai) = Dom(Bi) = Dom(Di) = \{0,...,9\} \\
            Dom(Ci) = \{0,1\}
        \end{gather*}
        $Ci$ steht für den Übertrag.  Man muss die Domänen der Summanden und
        des Ergebnisses von der des Übertrags trennen, da der Übertrag nur zwei
        verschiedene Werte annehmen kann. Für das Ergebnis der Addition gilt:
        \begin{align*}
            D1 &= (A1 + B1) \mod 10 \\
            Di &= (Ai + Bi + Ci) \mod 10 \text{ für } i = 2,3,4 \\
            D5 &= \lfloor (A4 + B4 + C4) / 10 \rfloor \\
            C2 &= \lfloor (A1 + B1) / 10 \rfloor \\
            C(i+1) &= \lfloor (Ai + Bi + Ci) / 10 \rfloor
            \text{ für }  i = 2,3
        \end{align*}

        \begin{landscape}
            \vspace*{\fill}
            \begin{figure}[h]
                \centering
                \begin{tikzpicture}[
                        auto,
                        scale=2.5,
                ]
                    \tikzstyle{edge}=[->,>=stealth']
                    \tikzstyle{var}=[circle, thick, draw, minimum size=6mm]
                    \tikzstyle{constraint}=[rectangle, thick, draw, align=center]

                    \node [var] (d1) at (8,0) {D1};
                    \node [var] (d2) at (6,0) {D2};
                    \node [var] (d3) at (4,0) {D3};
                    \node [var] (d4) at (2,0) {D4};
                    \node [var] (d5) at (0,0) {D5};
                    \node [var] (c2) at (6,1) {C2};
                    \node [var] (c3) at (4,1) {C3};
                    \node [var] (c4) at (2,1) {C4};
                    \node [var] (b1) at (8,2) {B1};
                    \node [var] (b2) at (6,2) {B2};
                    \node [var] (b3) at (4,2) {B3};
                    \node [var] (b4) at (2,2) {B4};
                    \node [var] (a1) at (8,3) {A1};
                    \node [var] (a2) at (6,3) {A2};
                    \node [var] (a3) at (4,3) {A3};
                    \node [var] (a4) at (2,3) {A4};

                    \node [constraint] (m1) at (7,0)
                        {$(A1 + B1)$ \\ $\mod 10 = D1$};
                    \node [constraint] (q1) at (7,3)
                        {$\lfloor(A1 + B1)$ \\ $/ 10 \rfloor = C2$};
                    \node [constraint] (m2) at (5,0)
                        {$(A2 + B2 + C2)$ \\ $\mod 10 = D2$};
                    \node [constraint] (q2) at (5,3)
                        {$\lfloor(A2 + B2 + C2)$ \\ $/ 10 \rfloor = C3$};
                    \node [constraint] (m3) at (3,0)
                        {$(A3 + B3 + C3)$ \\ $\mod 10 = D3$};
                    \node [constraint] (q3) at (3,3)
                        {$\lfloor(A3 + B3 + C3)$ \\ $/ 10 \rfloor = C4$};
                    \node [constraint] (m4) at (1,0)
                        {$(A4 + B4 + C4)$ \\ $\mod 10 = D4$};
                    \node [constraint] (q4) at (1,3)
                        {$\lfloor(A4 + B4 + C4)$ \\ $/ 10 \rfloor = D5$};

                    \draw (a1) -- (m1);
                    \draw (b1) -- (m1);
                    \draw (d1) -- (m1);
                    \draw (a1) -- (q1);
                    \draw (b1) -- (q1);
                    \draw (c2) -- (q1);

                    \draw (a2) -- (m2);
                    \draw (b2) -- (m2);
                    \draw (c2) -- (m2);
                    \draw (d2) -- (m2);
                    \draw (a2) -- (q2);
                    \draw (b2) -- (q2);
                    \draw (c2) -- (q2);
                    \draw (c3) -- (q2);

                    \draw (a3) -- (m3);
                    \draw (b3) -- (m3);
                    \draw (c3) -- (m3);
                    \draw (d3) -- (m3);
                    \draw (a3) -- (q3);
                    \draw (b3) -- (q3);
                    \draw (c3) -- (q3);
                    \draw (c4) -- (q3);

                    \draw (a4) -- (m4);
                    \draw (b4) -- (m4);
                    \draw (c4) -- (m4);
                    \draw (d4) -- (m4);
                    \draw (a4) -- (q4);
                    \draw (b4) -- (q4);
                    \draw (c4) -- (q4);
                    \draw (d5) -- (q4);
                \end{tikzpicture}
                \caption{Constraint Network}
                \label{fig:constraints}
            \end{figure}
            \vspace*{\fill}
        \end{landscape}
	   
    \item

    \item

\end{enumerate}


\end{document}
