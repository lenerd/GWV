\documentclass[a4paper]{scrartcl}

% font/encoding packages
\usepackage[utf8]{inputenc}
\usepackage[T1]{fontenc}
\usepackage{lmodern}
\usepackage[ngerman]{babel}
\usepackage[ngerman=ngerman-x-latest]{hyphsubst}

\usepackage{amsmath, amssymb, amsfonts, amsthm}
\usepackage{array}
\usepackage{stmaryrd}
\usepackage{marvosym}
\allowdisplaybreaks
\usepackage[output-decimal-marker={,}]{siunitx}
\usepackage[shortlabels]{enumitem}
\usepackage[section]{placeins}
\usepackage{float}
\usepackage{units}
\usepackage{listings}
\usepackage{pgfplots}
\pgfplotsset{compat=1.12}
\usepackage[hyphens]{url}
\usepackage{hyperref}
\usepackage{pdflscape}

\usepackage{tikz}
\usetikzlibrary{arrows,automata}
\usepackage{verbatim}

%\usepackage[newfloat]{minted}

\lstset{
    language=Python,
    numbers=left,
    frame=single,
    basicstyle=\footnotesize\ttfamily,
    otherkeywords={with,as},
}

\newtheorem*{behaupt}{Behauptung}
\newcommand{\gdw}{\Leftrightarrow}

\usepackage{fancyhdr}
\pagestyle{fancy}

\def \blattnr {6}

\lhead{GWV - Blatt {\blattnr}}
\rhead{Billis, Braun, Knapperzbusch, Nikolaisen}
\cfoot{\thepage}


\title{Grundlagen der Wissensverarbeitung}
\subtitle{Blatt {\blattnr} Hausaufgaben}
\author{
    Fabian Billis (6720351) \\
    Lennart Braun (6523742), \\
    Maximilian Knapperzbusch (6535090) \\
    Laurens Nikolaisen (6527179) \\
}
\date{zum 23. November 2015}

\begin{document}
\maketitle

\section*{Exercise \blattnr.1: Constraints}
\begin{enumerate}
    \item
        Mit Hilfe der Regeln der schriftlichen Addition, kann man das Problem
        formalisieren.
        \begin{center}
            \begin{tabular}{crrrrr}
                  &    & A4 & A3 & A2 & A1 \\
                  &    & B4 & B3 & B2 & B1 \\
                + & C5 & C4 & C3 & C2 & C1 \\ \hline
                  & D5 & D4 & D3 & D2 & D1
            \end{tabular}
        \end{center}
        Für die Domänen gilt:
        \begin{gather*}
            Dom(Ai) = Dom(Bi) = Dom(Di) = \{0,...,9\} \\
            Dom(Ci) = \{0,1\}
        \end{gather*}
        $Ci$ steht für den Übertrag.  Man muss die Domänen der Summanden und
        des Ergebnisses von der des Übertrags trennen, da der Übertrag nur zwei
        verschiedene Werte annehmen kann. Für das Ergebnis der Addition gilt:
        \begin{align*}
            D1 &= (A1 + B1) \mod 10 \\
            Di &= (Ai + Bi + Ci) \mod 10 \text{ für } i = 2,3,4 \\
            D5 &= \lfloor (A4 + B4 + C4) / 10 \rfloor \\
            C2 &= \lfloor (A1 + B1) / 10 \rfloor \\
            C(i+1) &= \lfloor (Ai + Bi + Ci) / 10 \rfloor
            \text{ für }  i = 2,3
        \end{align*}

        \begin{landscape}
            \vspace*{\fill}
            \begin{figure}[h]
                \centering
                \begin{tikzpicture}[
                        auto,
                        scale=2.5,
                ]
                    \tikzstyle{edge}=[->,>=stealth']
                    \tikzstyle{var}=[circle, thick, draw, minimum size=6mm]
                    \tikzstyle{constraint}=[rectangle, thick, draw, align=center]

                    \node [var] (d1) at (8,0) {D1};
                    \node [var] (d2) at (6,0) {D2};
                    \node [var] (d3) at (4,0) {D3};
                    \node [var] (d4) at (2,0) {D4};
                    \node [var] (d5) at (0,0) {D5};
                    \node [var] (c2) at (6,1) {C2};
                    \node [var] (c3) at (4,1) {C3};
                    \node [var] (c4) at (2,1) {C4};
                    \node [var] (b1) at (8,2) {B1};
                    \node [var] (b2) at (6,2) {B2};
                    \node [var] (b3) at (4,2) {B3};
                    \node [var] (b4) at (2,2) {B4};
                    \node [var] (a1) at (8,3) {A1};
                    \node [var] (a2) at (6,3) {A2};
                    \node [var] (a3) at (4,3) {A3};
                    \node [var] (a4) at (2,3) {A4};

                    \node [constraint] (m1) at (7,0)
                        {$(A1 + B1)$ \\ $\mod 10 = D1$};
                    \node [constraint] (q1) at (7,3)
                        {$\lfloor(A1 + B1)$ \\ $/ 10 \rfloor = C2$};
                    \node [constraint] (m2) at (5,0)
                        {$(A2 + B2 + C2)$ \\ $\mod 10 = D2$};
                    \node [constraint] (q2) at (5,3)
                        {$\lfloor(A2 + B2 + C2)$ \\ $/ 10 \rfloor = C3$};
                    \node [constraint] (m3) at (3,0)
                        {$(A3 + B3 + C3)$ \\ $\mod 10 = D3$};
                    \node [constraint] (q3) at (3,3)
                        {$\lfloor(A3 + B3 + C3)$ \\ $/ 10 \rfloor = C4$};
                    \node [constraint] (m4) at (1,0)
                        {$(A4 + B4 + C4)$ \\ $\mod 10 = D4$};
                    \node [constraint] (q4) at (1,3)
                        {$\lfloor(A4 + B4 + C4)$ \\ $/ 10 \rfloor = D5$};

                    \draw (a1) -- (m1);
                    \draw (b1) -- (m1);
                    \draw (d1) -- (m1);
                    \draw (a1) -- (q1);
                    \draw (b1) -- (q1);
                    \draw (c2) -- (q1);

                    \draw (a2) -- (m2);
                    \draw (b2) -- (m2);
                    \draw (c2) -- (m2);
                    \draw (d2) -- (m2);
                    \draw (a2) -- (q2);
                    \draw (b2) -- (q2);
                    \draw (c2) -- (q2);
                    \draw (c3) -- (q2);

                    \draw (a3) -- (m3);
                    \draw (b3) -- (m3);
                    \draw (c3) -- (m3);
                    \draw (d3) -- (m3);
                    \draw (a3) -- (q3);
                    \draw (b3) -- (q3);
                    \draw (c3) -- (q3);
                    \draw (c4) -- (q3);

                    \draw (a4) -- (m4);
                    \draw (b4) -- (m4);
                    \draw (c4) -- (m4);
                    \draw (d4) -- (m4);
                    \draw (a4) -- (q4);
                    \draw (b4) -- (q4);
                    \draw (c4) -- (q4);
                    \draw (d5) -- (q4);
                \end{tikzpicture}
                \caption{Constraint Network}
                \label{fig:constraints}
            \end{figure}
            \vspace*{\fill}
        \end{landscape}
	   
    \item
	Zunächst guckt man sich an, wie welche Wörte man verweden darf, um das Rätsel zu lösen. Danach
	muss man sich ansehen, welche Worte so miteinander kombiniert werden können, sodass das die Buchstaben
	eines Wortes auch Teil eines anderen seien können, damit die Regeln des Rätsels erfüllt werden. \\
	Nach all den Überlegungen, wählt man ein Wort aus, mit welchem man starten möchte. Sobald man dieses 
	dann gesetzt hat, prüft man wie man andere Worte an dieses anlegen kann. \\
	Das Problem bei diesem Verfahren ist, dass man nur schwer einen Überblick behalten kann und man viel
	voraus planen muss. Wenn das Wort, mit dem man beginnt falsch gewählt wurde, muss man mit Pech von
	ganz vorne anfangen. Ein weiteres Problem dieser Vorgehensweise besteht darin, dass man sich zu sehr 
	auf eine mögliche Lösung fixiert ist und sehr schwer einen neuen Ansatz finden kann. \\

    \item
        Wir repräsentieren das Problem durch sechs Variablen A1 bis D3, die
        jeweils die Worte in den Zeilen bzw. Spalten angeben. Die Domain einer
        Variablen ist jeweils die Menge der angegebenen Wörter.
        Die Constraints verlangen, dass der erste Buchstabe von A1 und D1
        gleich sein müssen etc. (siehe Abb. \ref{fig:constraints2})

        Da keine unären Constaints vorhanden sind, ist Domain Consistency
        bereits gegeben.

        Die Ausführung des Arc Consistency Algorithmus per Hand dauert mir zu
        lange.
        \begin{figure}[h]
            \centering
            \begin{tikzpicture}[
                    auto,
                    scale=1,
            ]
                \tikzstyle{edge}=[->,>=stealth']
                \tikzstyle{var}=[circle, thick, draw, minimum size=6mm]
                \tikzstyle{constraint}=[rectangle, thick, draw, align=center]

                \node [var] (A1) at (0,7) {A1};
                \node [var] (A2) at (0,4) {A2};
                \node [var] (A3) at (0,1) {A3};
                \node [var] (D1) at (10,7) {D1};
                \node [var] (D2) at (10,4) {D2};
                \node [var] (D3) at (10,1) {D3};

                \node [constraint] (c1) at (5,8) {A1[0] = D1[0]};
                \node [constraint] (c2) at (5,7) {A1[1] = D2[0]};
                \node [constraint] (c3) at (5,6) {A1[2] = D3[0]};
                \node [constraint] (c4) at (5,5) {A2[0] = D1[1]};
                \node [constraint] (c5) at (5,4) {A2[1] = D2[1]};
                \node [constraint] (c6) at (5,3) {A2[2] = D3[1]};
                \node [constraint] (c7) at (5,2) {A3[0] = D1[2]};
                \node [constraint] (c8) at (5,1) {A3[1] = D2[2]};
                \node [constraint] (c9) at (5,0) {A3[2] = D3[2]};

                \draw (A1) -- (c1.west);
                \draw (A1) -- (c2.west);
                \draw (A1) -- (c3.west);
                \draw (A2) -- (c4.west);
                \draw (A2) -- (c5.west);
                \draw (A2) -- (c6.west);
                \draw (A3) -- (c7.west);
                \draw (A3) -- (c8.west);
                \draw (A3) -- (c9.west);
                \draw (D1) -- (c1.east);
                \draw (D1) -- (c4.east);
                \draw (D1) -- (c7.east);
                \draw (D2) -- (c2.east);
                \draw (D2) -- (c5.east);
                \draw (D2) -- (c8.east);
                \draw (D3) -- (c3.east);
                \draw (D3) -- (c6.east);
                \draw (D3) -- (c9.east);

            \end{tikzpicture}
            \caption{Constraint Network}
            \label{fig:constraints2}
        \end{figure}

\end{enumerate}


\end{document}
