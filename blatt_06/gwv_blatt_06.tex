\documentclass[a4paper]{scrartcl}

% font/encoding packages
\usepackage[utf8]{inputenc}
\usepackage[T1]{fontenc}
\usepackage{lmodern}
\usepackage[ngerman]{babel}
\usepackage[ngerman=ngerman-x-latest]{hyphsubst}

\usepackage{amsmath, amssymb, amsfonts, amsthm}
\usepackage{array}
\usepackage{stmaryrd}
\usepackage{marvosym}
\allowdisplaybreaks
\usepackage[output-decimal-marker={,}]{siunitx}
\usepackage[shortlabels]{enumitem}
\usepackage[section]{placeins}
\usepackage{float}
\usepackage{units}
\usepackage{listings}
\usepackage{pgfplots}
\pgfplotsset{compat=1.12}
\usepackage[hyphens]{url}
\usepackage{hyperref}

\usepackage{tikz}
\usetikzlibrary{arrows,automata}
\usepackage{verbatim}

%\usepackage[newfloat]{minted}

\lstset{
    language=Python,
    numbers=left,
    frame=single,
    basicstyle=\footnotesize\ttfamily,
    otherkeywords={with,as},
}

\newtheorem*{behaupt}{Behauptung}
\newcommand{\gdw}{\Leftrightarrow}

\usepackage{fancyhdr}
\pagestyle{fancy}

\def \blattnr {6}

\lhead{GWV - Blatt {\blattnr}}
\rhead{Billis, Braun, Knapperzbusch, Nikolaisen}
\cfoot{\thepage}


\title{Grundlagen der Wissensverarbeitung}
\subtitle{Blatt {\blattnr} Hausaufgaben}
\author{
    Fabian Billis (6720351) \\
    Lennart Braun (6523742), \\
    Maximilian Knapperzbusch (6535090) \\
    Laurens Nikolaisen (6527179) \\
}
\date{zum 23. November 2015}

\begin{document}
\maketitle

\section*{Exercise \blattnr.1: Constraints}
\begin{enumerate}
	\item
	Mit Hilfe der Regeln der schriftlichen Addition, kann man das Problem formalisieren. \\
	\begin{center}
		\begin{tabular}{cr}
				&A4 A3 A2 A1 \\
				&B4 B3 B2 B1 \\
				+& C4 C3 C2 C1    \\\hline
				&D5 D4 D3 D2 D1   
		\end{tabular}
	\end{center}
	Für die Domäne gilt: \\
	\begin{center}
	$Dom(Ai) = Dom(Bi) = Dom(Di) = \{0,...,9\}$ \\
	$Dom(Ci) = \{0,1\}$ \\
	\end{center}
	Man muss die Domänen der Summanden und des Ergebnisses von der des Übertrags trennen, da der Übertrag 
	nur zwei verschiedene Werte annehmen kann. Für das Ergebnis der Addition gilt: \\
	\begin{center}
	$D1 = (A1 + B1)\mod b$ \\
	$Di = (Ai + Bi + Ci)\mod b$  für $i = 2,3,4$\\
	$D5 = C5$ \\
	$Ci = \lfloor(Ai + Bi + Ci) / b\rfloor$ für $i = 1,2,3,4$ \\
	\end{center}
	$b$ steht für die Basis des Zahlensystems, mit dem gerechnet wird, z.B. binär, dezimal oder 
	hexadezimal. Diese Modulorechnung hat zur Folge, dass nur einstellige Werte in der Rechnung auftreten.
	
	
	
	   
			
\end{enumerate}


\end{document}
